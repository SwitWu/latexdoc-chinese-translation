% !TEX program = xelatex
\documentclass{ltxdockit}[2010/09/26]
\usepackage{ctex}
\defcaptionname{\languagename}{\contentsname}{\centerline{目\,录}}
\makeatletter
\define@key{ltd@ttp}{date}{\def\ltd@title@date{#1\par 翻译\par\zhtoday}}
\define@key{ltd@ttp}{translator}{\def\ltd@title@translator{#1}}
\patchcmd{\printtitlepage}{Version}
         {\raggedleft\ltd@title@translator\par
          翻译日期\par\raggedright
          版本号}{}{}
\makeatother
\usepackage[utf8]{inputenc}
\usepackage[american]{babel}
\usepackage[strict]{csquotes}
\usepackage{shortvrb}
\MakeAutoQuote*{<}{>}
\MakeShortVerb{\|}

\titlepage{%
  title={\sty{etoolbox} 宏包},
  subtitle={面向类和包作者的 \etex 工具箱},
  url={http://www.ctan.org/pkg/etoolbox/},
  author={Philipp Lehman, Joseph Wright},
  email={joseph.wright@morningstar2.co.uk},
  revision={v2.5k},
  date={2020年10月5日},
  translator={张泓知}}

\hypersetup{
  pdftitle={etoolbox宏包},
  pdfsubject={面向类和包作者的e-TeX工具箱},
  pdfauthor={Philipp Lehman, Joseph Wright},
  pdfkeywords={tex, e-tex, latex, class, package, programming}}

\begin{document}

\printtitlepage
\tableofcontents

\section{介绍}
\label{int}

\subsection[关于]{关于 \sty{etoolbox}}

\sty{etoolbox} 宏包是一个编程工具箱,主要面向 \LaTeX 类和宏包作者。它提供了
一些 \etex 提供的新原语的 \LaTeX 前端,以及一些与 \etex 无关但与本宏包的特
性相匹配的通用工具。

\subsection{许可证}

版权 \textcopyright\ 2007--2011 Philipp Lehman,2015--2020 Joseph Wright。根
据 \lppl \fnurl{http://www.latex-project.org/lppl/} 第 1.3c 版或更高版本的条款,
允许复制、分发和/或修改此软件。


\section{用户命令}
\label{use}

本节中的工具旨在面向常规用户以及类和宏包作者。

\subsection{定义}
\label{use:def}

\begin{ltxsyntax}

  \cmditem{newrobustcmd}{command}[arguments][optarg default]{replacement text}
  \cmditem*{newrobustcmd*}{command}[arguments][optarg default]{replacement text}

  此命令的语法和行为类似于 \cmd{newcommand},但新定义的 \prm{command}
  将是健壮的 (robust)。此命令的行为与 \LaTeX 核心的 \cmd{DeclareRobustCommand}
  命令不同之处在于,如果 \prm{command} 已经定义,它将发出错误消息而不仅仅是
  信息性消息。由于它使用 \etex 的低级保护机制而不是相应的更高级别的 \LaTeX
  工具,因此不需要额外的宏来实现<健壮性 (robustness) >。

  \cmditem{renewrobustcmd}{command}[arguments][optarg default]{replacement text}
  \cmditem*{renewrobustcmd*}{command}[arguments][optarg default]{replacement text}

  此命令的语法和行为类似于 \cmd{renewcommand},但重新定义的 \prm{command}
  将是健壮的 (robust)。

  \cmditem{providerobustcmd}{command}[arguments][optarg default]{replacement text}
  \cmditem*{providerobustcmd*}{command}[arguments][optarg default]{replacement text}

  此命令的语法和行为类似于 \cmd{providecommand},但新定义的 \prm{command}
  将是健壮的 (robust)。请注意,此命令只会为未定义的 \prm{command} 提供健壮 (robust) 定义。它
  不会使已经定义的 \prm{command} 变得健壮 (robust)。
\end{ltxsyntax}

\subsection{补丁 (Patching)}
\label{use:pat}

\begin{ltxsyntax}

  \cmditem{robustify}{command}

  重新定义了一个由 \cmd{newcommand} 定义的 \prm{command},使其成为健壮 (robust)命令,
  而不更改其参数、前缀或替换文本。如果 \prm{command} 是用 \cmd{DeclareRobustCommand}
  定义的,则会自动检测到,并且 \LaTeX 的高级保护机制将被相应的低级 \etex 功能
  替换。

\end{ltxsyntax}

\subsection{保护}
\label{use:pro}

\begin{ltxsyntax}

  \cmditem{protecting}{code}

  此命令将 \LaTeX 的保护机制(通常需要在每个易损命令前加上 \cmd{protect})应用于任
  意 \prm{code} 的整个代码块。其行为取决于当前的 \cmd{protect} 状态。请注意,
  即使是单个记号,\prm{code} 周围的大括号也是必需的。

\end{ltxsyntax}

\subsection[长度与计数器]{长度与计数器分配}
\label{use:cal}

此部分的工具是 \cmd{setcounter} 和 \cmd{setlength} 的替代品,支持算术表达式。

\begin{ltxsyntax}

  \cmditem{defcounter}{counter}{integer expression}

  将值分配给之前用 \cmd{newcounter} 初始化的 \LaTeX \prm{counter}。此命令在概念
  和语法上类似于 \cmd{setcounter},但有两个主要区别。1)第二个参数可以是一个
  \prm{integer expression},它将使用 \cmd{numexpr} 处理。 \prm{integer
    expression} 可以是此上下文中有效的任意代码。分配给 \prm{counter} 的值将是
  计算结果。2)与 \cmd{setcounter} 相反,默认情况下,此赋值是局部的,但
  \cmd{defcounter} 可以使用 \cs{global} 进行前缀。 \cmd{setcounter} 的功能
  等效于 \cs{global}\cmd{defcounter}。

  \cmditem{deflength}{length}{glue expression}

  将值分配给之前用 \cmd{newlength} 初始化的 \prm{length} 寄存器。此命令在概念
  和语法上类似于 \cmd{setlength},但第二个参数可以是一个 \prm{glue expression},
  它将使用 \cmd{glueexpr} 处理。 \prm{glue expression} 可以是此上下文中有效的任
  意代码。分配给 \prm{length} 寄存器的值将是计算结果。此赋值默认情况下是局部
  的,但 \cmd{deflength} 可以使用 \cs{global} 进行前缀。此命令可以作为
  \cmd{setlength} 的替代。

\end{ltxsyntax}

\subsection[文档钩子]{附加文档钩子}
\label{use:pre}

\LaTeX 提供了两个钩子,可以推迟代码的执行,要么推迟到文档正文的开头,要么推迟到文档的末尾。任何 \cmd{AtBeginDocument}
代码都会在主 \file{aux} 文件首次读取后,在文档正文的开头执行。任何 \cmd{AtEndDocument} 代码都会在文档末尾执行,在主
\file{aux} 文件第二次读取之前执行。这里介绍的钩子在概念上类似,但将其 \prm{code} 参数的执行推迟到略微不同的位置。
\prm{code} 可以是任意 \TeX 代码。 \prm{code} 参数中的参数字符是允许的,不需要加倍。

\begin{ltxsyntax}

\cmditem{AfterPreamble}{code}

此钩子是 \cmd{AtBeginDocument} 的变体,可在导言部分和文档正文中使用。在导言部分使用时,其行为与 \cmd{AtBeginDocument}
完全相同。在文档正文中使用时,它会立即执行其 \prm{code} 参数。 \cmd{AtBeginDocument} 在这种情况下会发出错误。此钩子非常
有用,用于推迟需要写入主 \file{aux} 文件的代码。

\cmditem{AtEndPreamble}{code}

此钩子与 \cmd{AtBeginDocument} 不同,因为 \prm{code} 在导言部分的最后执行,在主 \file{aux} 文件(在前一个 \LaTeX\ pass
上写入)被读取之前,也在任何 \cmd{AtBeginDocument} 代码之前执行。请注意,此时不可能向 \file{aux} 文件写入。

\cmditem{AfterEndPreamble}{code}

此钩子与 \cmd{AtBeginDocument} 不同,因为 \prm{code} 在 |\begin{document}| 的最后执行,而且在任何 \cmd{AtBeginDocument}
代码之后执行。请注意,在此钩子执行时,使用\hfil\break \cmd{@onlypreamble} 限定其作用域的命令将不再可用。

\cmditem{AfterEndDocument}{code}

此钩子与 \cmd{AtEndDocument} 不同,因为 \prm{code} 在文档的最后执行,在当前 \LaTeX pass 上写入的主 \file{aux} 文件被读取
之后,并在任何 \cmd{AtEndDocument} 代码之后执行。

\end{ltxsyntax}

从某种意义上说,\cmd{AtBeginDocument} 代码既不是导言部分的一部分,也不是文档正文的一部分,因为它在排版之前执行初始化序列的
中间位置。有时,将代码移至导言部分的末尾是有必要的,因为此时已经加载了所有请求的包。然而,如果代码需要在 \file{aux} 文件中
执行,\cmd{AtBeginDocument} 代码执行太晚了。相反,\cmd{AtEndPreamble} 代码是导言部分的一部分;\cmd{AfterEndPreamble}
代码是文档正文的一部分,并可能包含要在文档开头排版的可打印文本。总而言之,在 |\begin{document}| 中,\LaTeX 将执行以下任务:

\begin{itemize}
  \setlength{\itemsep}{0pt}
  \item 执行任何 \cmd{AtEndPreamble} 代码
  \item 开始文档正文的初始化(页面布局、默认字体等)
  \item 加载前一个 \LaTeX pass 上写入的主 \file{aux} 文件
  \item 在当前 pass 上打开主 \file{aux} 文件进行写入
  \item 继续文档正文的初始化
  \item 执行任何 \cmd{AtBeginDocument} 代码
  \item 完成文档正文的初始化
  \item 禁用所有 \cmd{@onlypreamble} 命令
  \item 执行任何 \cmd{AfterEndPreamble} 代码
\end{itemize}
%
在 |\end{document}| 中,\LaTeX 将执行以下任务:

\begin{itemize}
  \setlength{\itemsep}{0pt}
  \item 执行任何 \cmd{AtEndDocument} 代码
  \item 执行最终的 \cmd{clearpage} 操作
  \item 关闭写入的主 \file{aux} 文件
  \item 加载当前 pass 上写入的主 \file{aux} 文件
  \item 执行最终的测试和警告,如果适用的话
  \item 执行任何 \cmd{AfterEndDocument} 代码
\end{itemize}
%
任何 \cmd{AtEndDocument} 代码都可以被视为文档正文的一部分,因为在其执行时,仍然可以执行排版任务并写入主 \file{aux} 文件。
\cmd{AfterEndDocument} 代码不是文档正文的一部分。此钩子对于在 \LaTeX pass 的最后评估 \file{aux} 文件中的数据非常有用。

\subsection[环境钩子]{环境钩子}
\label{use:env}

本节工具提供了针对环境的钩子。请注意,它们不会修改\prm{environment}的定义。它们挂钩到 \cmd{begin} 和 \cmd{end} 命令上。
重新定义 \prm{environment} 不会清除相应的钩子。参数字符在 \prm{code} 参数中是允许的,无需加倍。

\begin{ltxsyntax}

  \cmditem{AtBeginEnvironment}{environment}{code}

  将任意 \prm{code} 附加到由 \cmd{begin} 命令在给定 \prm{environment} 的开头执行的钩子中,在 \cmd{\prm{environment}}
  之前,位于由 \cmd{begin} 打开的组内。

  \cmditem{AtEndEnvironment}{environment}{code}

  将任意 \prm{code} 附加到由 \cmd{end} 命令在给定 \prm{environment} 的末尾执行的钩子中,在 \cmd{end\prm{environment}}
  之前,位于由 \cmd{begin} 打开的组内。

  \cmditem{BeforeBeginEnvironment}{environment}{code}

  将任意 \prm{code} 附加到由 \cmd{begin} 命令在非常早期执行的钩子中,在持有环境的组被打开之前。

  \cmditem{AfterEndEnvironment}{environment}{code}

  将任意 \prm{code} 附加到由 \cmd{end} 命令在非常晚期执行的钩子中,在持有环境的组被关闭后执行。

\end{ltxsyntax}

\section{作者命令}

本节工具面向类和包的作者。

\subsection{定义}

\subsubsection{宏定义}
\label{aut:def:def}

本节工具是简单但经常需要的快捷方式,扩展了 \LaTeX 核心中 \cmd{@namedef} 和\hfil\break \cmd{@nameuse} 宏的作用范围。

\begin{ltxsyntax}

  \cmditem{csdef}{csname}<arguments>{replacement text}

  类似于 \TeX 原语 \cmd{def},但它以控制序列名称作为第一个参数。此命令是健壮的 (robust),对应于 \cmd{@namedef}。

  \cmditem{csgdef}{csname}<arguments>{replacement text}

  类似于 \TeX 原语 \cmd{gdef},但它以控制序列名称作为第一个参数。此命令是健壮的 (robust)。

  \cmditem{csedef}{csname}<arguments>{replacement text}

  类似于 \TeX 原语 \cmd{edef},但它以控制序列名称作为第一个参数。此命令是健壮的 (robust)。

  \cmditem{csxdef}{csname}<arguments>{replacement text}

  类似于 \TeX 原语 \cmd{xdef},但它以控制序列名称作为第一个参数。此命令是健壮的 (robust)。

  \cmditem{protected@csedef}{csname}<arguments>{replacement text}

  类似于 \cmd{csedef},但临时启用了 \LaTeX 的保护机制。换句话说,此命令类似于 \LaTeX 核心命令
  \cmd{protected@edef},但它以控制序列名称作为第一个参数。此命令是健壮的 (robust)。

  \cmditem{protected@csxdef}{csname}<arguments>{replacement text}

  类似于 \cmd{csxdef},但临时启用了 \LaTeX 的保护机制。换句话说,此命令类似于 \LaTeX 核心命令
  \cmd{protected@xdef},但它以控制序列名称作为第一个参数。此命令是健壮的 (robust)。

  \cmditem{cslet}{csname}{command}

  类似于 \TeX 原语 \cmd{let},但第一个参数是一个控制序列名称。如果 \prm{command} 未定义,在分配后
  \prm{csname} 也将被未定义。此命令是健壮的 (robust),可带有 \cs{global} 前缀。

  \cmditem{letcs}{command}{csname}

  类似于 \TeX 原语 \cmd{let},但第二个参数是一个控制序列名称。如果 \prm{csname} 未定义,在分配后
  \prm{command} 也将被未定义。此命令是健壮的 (robust),可带有 \cs{global} 前缀。

  \cmditem{csletcs}{csname}{csname}

  类似于 \TeX 原语 \cmd{let},但两个参数都是控制序列名称。如果第二个 \prm{csname} 未定义,在分配后第一个
  \prm{csname} 也将被未定义。此命令是健壮的 (robust),可带有 \cs{global} 前缀。

  \cmditem{csuse}{csname}

  以控制序列名称为参数,并形成控制序列记号。此命令不同于 \LaTeX 核心中的\hfil\break \cmd{@nameuse} 宏,如果控制
  序列未定义,它会展开为空字符串。

  \cmditem{undef}<command>

  清除 \prm{command},使得 \etex 的 \cmd{ifdefined} 和 \cmd{ifcsname} 测试将其视为未定义。此命令是
  健壮的 (robust),可带有 \cs{global} 前缀。

  \cmditem{gundef}<command>

  类似于 \cmd{undef},但是作用于全局。

  \cmditem{csundef}{csname}

  类似于 \cmd{undef},但它以控制序列名称作为参数。此命令是健壮的 (robust),可带有 \cs{global} 前缀。

  \cmditem{csgundef}{csname}

  类似于 \cmd{csundef},但是作用于全局。

  \cmditem{csmeaning}{csname}

  类似于 \TeX 原语 \cmd{meaning},但以控制序列名称作为参数。如果控制序列未定义,此命令不会隐式地将其赋予
  \cmd{relax} 的含义。

  \cmditem{csshow}{csname}

  类似于 \TeX 原语 \cmd{show},但以控制序列名称作为参数。如果控制序列未定义,此命令不会隐式地将其赋予
  \cmd{relax} 的含义。此命令是健壮的 (robust)。

\end{ltxsyntax}

\subsubsection{算术定义}
\label{aut:def:cal}

本节中的工具允许使用宏进行计算,而不是长度寄存器和计数器。

\begin{ltxsyntax}

  \cmditem{numdef}<command>{integer expression}

  类似于 \cmd{edef},不同之处在于 \prm{integer expression} 使用 \cmd{numexpr} 处理。\hfil\break \prm{integer expression}
  可以是在此上下文中有效的任意代码。分配给 \prm{command} 的替换文本将是该计算的结果。如果 \prm{command}
  未定义,则在处理 \prm{integer expression} 之前将其初始化为 \texttt{0}.

  \cmditem{numgdef}<command>{integer expression}

  类似于 \cmd{numdef},不同之处在于分配是全局的。

  \cmditem{csnumdef}{csname}{integer expression}

  类似于 \cmd{numdef},不同之处在于它以控制序列名称作为其第一个参数。

  \cmditem{csnumgdef}{csname}{integer expression}

  类似于 \cmd{numgdef},不同之处在于它以控制序列名称作为其第一个参数。

  \cmditem{dimdef}<command>{dimen expression}

  类似于 \cmd{edef} ,不同之处在于 \prm{dimen expression} 使用 \cmd{dimexpr} 处理。\hfil\break \prm{dimen expression}
  可以是在此上下文中有效的任意代码。分配给 \prm{command} 的替换文本将是该计算的结果。如果 \prm{command}
  未定义,则在处理 \prm{dimen expression} 之前将其初始化为 \texttt{0pt}。

  \cmditem{dimgdef}<command>{dimen expression}

  类似于 \cmd{dimdef},不同之处在于分配是全局的。

  \cmditem{csdimdef}{csname}{dimen expression}

  类似于 \cmd{dimdef},不同之处在于它以控制序列名称作为其第一个参数。

  \cmditem{csdimgdef}{csname}{dimen expression}

  类似于 \cmd{dimgdef},不同之处在于它以控制序列名称作为其第一个参数。

  \cmditem{gluedef}<command>{glue expression}

  类似于 \cmd{edef},不同之处在于 \prm{glue expression} 使用 \cmd{glueexpr} 处理。 \prm{glue expression}
  可以是在此上下文中有效的任意代码。分配给 \prm{command} 的替换文本将是该计算的结果。如果
  \prm{command} 未定义,则在处理 \prm{glue expression} 之前将其初始化为 \texttt{0pt plus 0pt minus 0pt}。

  \cmditem{gluegdef}<command>{glue expression}

  类似于 \cmd{gluedef},不同之处在于分配是全局的。

  \cmditem{csgluedef}{csname}{glue expression}

  类似于 \cmd{gluedef},不同之处在于它以控制序列名称作为其第一个参数。

  \cmditem{csgluegdef}{csname}{glue expression}

  类似于 \cmd{gluegdef},不同之处在于它以控制序列名称作为其第一个参数。

  \cmditem{mudef}<command>{muglue expression}

  类似于 \cmd{edef} ,不同之处在于 \prm{muglue expression} 使用 \cmd{muexpr} 处理。\hfil\break \prm{muglue expression}
  可以是在此上下文中有效的任意代码。分配给 \prm{command} 的替换文本将是该计算的结果。如果 \prm{command}
  未定义,则在处理 \prm{muglue expression} 之前将其初始化为 \texttt{0mu}。

  \cmditem{mugdef}<command>{muglue expression}

  类似于 \cmd{mudef},不同之处在于分配是全局的。

  \cmditem{csmudef}{csname}{muglue expression}

  类似于 \cmd{mudef},不同之处在于它以控制序列名称作为其第一个参数。

  \cmditem{csmugdef}{csname}{muglue expression}

  类似于 \cmd{mugdef},不同之处在于它以控制序列名称作为其第一个参数。

\end{ltxsyntax}

\subsection{展开控制}
\label{aut:exp}

本节中的工具对于控制在 \cmd{edef} 或类似情境下的展开很有用。

\begin{ltxsyntax}

  \cmditem{expandonce}<command>

  此命令展开一次 \prm{command} 并阻止进一步展开替换文本。此命令可展开。

  \cmditem{csexpandonce}{csname}

  类似于 \cmd{expandonce},不同之处在于它以控制序列名称作为其参数。

\end{ltxsyntax}

\subsection{钩子管理}
\label{aut:hok}

本节中的工具旨在进行钩子管理。在此上下文中,\prm{hook} 是一个没有任何参数和前缀的普通宏,用于收集稍后要执行的代码。
这些工具也可以用于通过将代码附加到其替换文本来修补简单的宏。对于更复杂的修补操作,请参见第 \ref{aut:pat} 节。本节
中的所有命令都将初始化 \prm{hook},如果未定义的话。

\subsubsection{向钩子追加内容}
\label{aut:hok:app}

本节中的工具用于向钩子追加任意代码。

\begin{ltxsyntax}

  \cmditem{appto}<hook>{code}

  此命令将任意 \prm{code} 追加到 \prm{hook}。如果 \prm{code} 包含任何参数字符,它们无需加倍。此命令是健壮的 (robust)。

  \cmditem{gappto}<hook>{code}

  类似于 \cmd{appto},不同之处在于分配是全局的。此命令可用作 \LaTeX 核心中的\hfil\break \cmd{g@addto@macro} 命令的替代品。

  \cmditem{eappto}<hook>{code}

  此命令将任意 \prm{code} 追加到 \prm{hook}。在定义时展开 \prm{code}。仅展开新的 \prm{code},不展开
  \prm{hook} 的当前替换文本。此命令是健壮的 (robust)。

  \cmditem{xappto}<hook>{code}

  类似于 \cmd{eappto},不同之处在于分配是全局的。

  \cmditem{protected@eappto}<hook>{code}

  类似于 \cmd{eappto},不同之处在于暂时启用了 \LaTeX 的保护机制。

  \cmditem{protected@xappto}<hook>{code}

  类似于 \cmd{xappto},不同之处在于暂时启用了 \LaTeX 的保护机制。

  \cmditem{csappto}{csname}{code}

  类似于 \cmd{appto},不同之处在于它以控制序列名称作为其第一个参数。

  \cmditem{csgappto}{csname}{code}

  类似于 \cmd{gappto},不同之处在于它以控制序列名称作为其第一个参数。

  \cmditem{cseappto}{csname}{code}

  类似于 \cmd{eappto},不同之处在于它以控制序列名称作为其第一个参数。

  \cmditem{csxappto}{csname}{code}

  类似于 \cmd{xappto},不同之处在于它以控制序列名称作为其第一个参数。

  \cmditem{protected@cseappto}<hook>{code}

  类似于 \cmd{protected@eappto},不同之处在于它以控制序列名称作为其第一个参数。

  \cmditem{protected@csxappto}<hook>{code}

  类似于 \cmd{protected@xappto},不同之处在于它以控制序列名称作为其第一个参数。

\end{ltxsyntax}

\subsubsection{在钩子前面插入内容}
\label{aut:hok:pre}

本节中的工具向钩子前面插入任意代码,即将代码插入到钩子的开头而不是添加到末尾。

\begin{ltxsyntax}

  \cmditem{preto}<hook>{code}

  类似于 \cmd{appto},不同之处在于 \prm{code} 被前置。

  \cmditem{gpreto}<hook>{code}

  类似于 \cmd{preto},不同之处在于分配是全局的。

  \cmditem{epreto}<hook>{code}

  类似于 \cmd{eappto},不同之处在于 \prm{code} 被前置。

  \cmditem{xpreto}<hook>{code}

  类似于 \cmd{epreto},不同之处在于分配是全局的。

  \cmditem{protected@epreto}<hook>{code}

  类似于 \cmd{epreto},不同之处在于暂时启用了 \LaTeX 的保护机制。

  \cmditem{protected@xpreto}<hook>{code}

  类似于 \cmd{xpreto},不同之处在于暂时启用了 \LaTeX 的保护机制。

  \cmditem{cspreto}{csname}{code}

  类似于 \cmd{preto},不同之处在于它以控制序列名称作为其第一个参数。

  \cmditem{csgpreto}{csname}{code}

  类似于 \cmd{gpreto},不同之处在于它以控制序列名称作为其第一个参数。

  \cmditem{csepreto}{csname}{code}

  类似于 \cmd{epreto},不同之处在于它以控制序列名称作为其第一个参数。

  \cmditem{csxpreto}{csname}{code}

  类似于 \cmd{xpreto},不同之处在于它以控制序列名称作为其第一个参数。

  \cmditem{protected@csepreto}<hook>{code}

  类似于 \cmd{protected@epreto},不同之处在于它以控制序列名称作为其第一个参数。

  \cmditem{protected@csxpreto}<hook>{code}

  类似于 \cmd{protected@xpreto},不同之处在于它以控制序列名称作为其第一个参数。

\end{ltxsyntax}

\subsection{补丁 (Patching)}
\label{aut:pat}

本节中的工具用于钩取或修改现有代码。这里介绍的所有命令都保留了被修补命令的参数和前缀。请注意,\cs{outer}
命令可能无法被修补。另外,本节中的命令在修补失败时不会自动发出任何错误信息。相反,它们接受一个 \prm{failure}
参数,应该提供合适的回退代码或错误信息。在导言区使用 \cmd{tracingpatches} 可以让这些命令将调试信息写入传输文件。

\begin{ltxsyntax}

  \cmditem{patchcmd}[prefix]{command}{search}{replace}{success}{failure}

  此命令提取 \prm{command} 的替换文本,用 \prm{replace} 替换 \prm{search},然后重新组装\hfil\break \prm{command}。
  匹配模式与类别码无关,在被修补的 \prm{command} 的替换文本中匹配第一个 \prm{search} 模式。请注意,修补过程
  涉及对 \prm{command} 的替换文本进行解记号化,然后在修补后重新记号化为当前的类别码制度。@ 符号的类别码临时
  设置为 11。如果 \prm{command} 的替换文本中包含任何具有非标准类别码的记号,必须在修补之前调整相应的类别码。
  如果要替换或插入的代码涉及到 \prm{command} 的参数,则参数字符无需加倍。如果指定了可选的 \prm{prefix},它将
  替换 \prm{command} 的前缀。空的 \prm{prefix} 参数会从 \prm{command} 中删除所有前缀。此命令是局部的。此命令
  在修补之前隐式执行相当于 \cmd{ifpatchable} 的测试。如果此测试成功,则命令应用修补并执行 \prm{success}。如果
  测试失败,则执行 \prm{failure},但不修改原始的 \prm{command}。此命令是健壮的 (robust)。

  \cmditem{ifpatchable}{command}{search}{true}{false}

  如果可以使用 \cmd{patchcmd} 修补 \prm{command} 并在其替换文本中找到 \prm{search} 模式,则执行 \prm{true};
  否则执行 \prm{false}。此命令是健壮的 (robust)。

  \cmditem*{ifpatchable*}{command}{true}{false}

  类似于 \cmd{ifpatchable},但星号变体不需要搜索模式。使用此版本检查是否可以使用 \cmd{apptocmd} 和 \cmd{pretocmd}
  修补命令。

  \cmditem{apptocmd}{command}{code}{success}{failure}

  此命令向 \prm{command} 的替换文本追加 \prm{code}。如果 \prm{command} 是一个无参数的宏,则其行为类似于第
  \ref{aut:hok:app} 节中的 \cmd{appto}。与 \cmd{appto} 不同,\cmd{apptocmd} 还可以用于修补带参数的命令。在这种
  情况下,它将解记号化 \prm{command} 的替换文本,应用修补,并在当前类别码制度下重新记号化。@ 符号的类别码临时设置为
  11。 \prm{code} 可能涉及 \prm{command} 的参数。此命令是局部的。如果修补成功,则执行 \prm{success}。如果修补失
  败,则执行 \prm{failure},但不修改原始的 \prm{command}。此命令是健壮的 (robust)。

  \cmditem{pretocmd}{command}{code}{success}{failure}

  此命令与 \cmd{apptocmd} 类似,但 \prm{code} 被插入到 \prm{command} 的替换文本开头。如果 \prm{command} 是一个
  无参数的宏,则其行为类似于第 \ref{aut:hok:app} 节中的 \cmd{preto}。与 \cmd{preto} 不同,\cmd{pretocmd} 也可用
  于修补带参数的命令。在这种情况下,它将解记号化 \prm{command} 的替换文本,应用修补,并在当前类别码制度下重新记号化。
  @ 符号的类别码临时设置为 11。 \prm{code} 可能涉及 \prm{command} 的参数。此命令是局部的。如果修补成功,则执行
  \prm{success}。如果修补失败,则执行 \prm{failure},但不修改原始的 \prm{command}。此命令是健壮的 (robust)。

  \csitem{tracingpatches}

  启用所有修补命令的跟踪,包括 \cmd{ifpatchable}。调试信息将写入传输文件。如果无法确定为何不应用修补或 \cmd{ifpatchable}
  返回 \prm{false},则此命令很有用。此命令必须在导言区使用。

\end{ltxsyntax}

\subsection{布尔标志}
\label{aut:bol}

该包提供了两个完全独立的布尔标志接口。第 \ref{aut:bo1:bol} 节中的工具是 \cmd{newif} 的 \LaTeX 前端。第
\ref{aut:bo1:tgl} 节中的工具使用了不同的机制。

\subsubsection{\TeX 标志}
\label{aut:bo1:bol}

由于本节中的工具在内部基于 \cmd{newif},因此它们可以用于测试和更改先前用\hfil\break \cmd{newif} 定义的标志的状态。它们也与
\sty{ifthen} 包的布尔测试兼容,并可以作为查询 \TeX 原语(如 \cmd{ifmmode})的 \LaTeX 接口。使用 \cmd{newif}
方法需要每个标志总共三个宏。

\begin{ltxsyntax}

  \cmditem{newbool}{name}

  定义名为 \prm{name} 的新布尔标志。如果已经定义了该标志,则此命令会发出错误。新定义的标志的初始状态为
  \texttt{false}。此命令是健壮的 (robust)。

  \cmditem{providebool}{name}

  定义名为 \prm{name} 的新布尔标志,除非已经定义。此命令是健壮的 (robust)。

  \cmditem{booltrue}{name}

  将布尔标志 \prm{name} 设置为 \texttt{true}。此命令是健壮的 (robust),可以使用 \cs{global} 修饰。如果标志未定义,
  它将发出错误。

  \cmditem{boolfalse}{name}

  将布尔标志 \prm{name} 设置为 \texttt{false}。此命令是健壮的 (robust),可以使用 \cs{global} 修饰。如果标志未定义,
  它将发出错误。

  \cmditem{setbool}{name}{value}

  将布尔标志 \prm{name} 设置为 \prm{value},可以是 \texttt{true} 或 \texttt{false}。此命令是健壮的 (robust),可以
  使用 \cs{global} 修饰。如果标志未定义,它将发出错误。

  \cmditem{ifbool}{name}{true}{false}

  如果布尔标志 \prm{name} 的状态为 \texttt{true},则展开为 \prm{true},否则为 \prm{false}。如果标志未定义,
  此命令将发出错误。此命令可用于执行基于普通 \TeX 语法的任何布尔测试,即通常使用以下方式:

  \begin{ltxcode}
    <<\iftest>> true<<\else>> false<<\fi>>
  \end{ltxcode}

  这包括所有使用 \cmd{newif} 定义的标志,以及 \TeX 原语,如 \cmd{ifmmode}。在表达式中使用标志或原语时省略
  \cmd{if} 前缀。例如:

  \begin{ltxcode}
    <<\ifmytest>> true\else false\fi
    <<\ifmmode>> true\else false\fi
  \end{ltxcode}
  %
  变为

  \begin{ltxcode}
    \ifbool{<<mytest>>}{true}{false}
    \ifbool{<<mmode>>}{true}{false}
  \end{ltxcode}

  \cmditem{notbool}{name}{not true}{not false}

  类似于 \cmd{ifbool},但对测试进行否定。

\end{ltxsyntax}

\subsubsection{\LaTeX 标志}
\label{aut:bo1:tgl}

与第 \ref{aut:bo1:bol} 节中的标志相比,本节中的工具每个标志仅需要一个宏。它们还使用单独的命名空间,以避免
与常规宏产生名称冲突。

\begin{ltxsyntax}

  \cmditem{newtoggle}{name}

  定义名为 \prm{name} 的新布尔标志。如果已经定义了该标志,则此命令会发出错误。新定义的标志的初始状态为
  \texttt{false}。此命令是健壮的 (robust)。

  \cmditem{providetoggle}{name}

  定义名为 \prm{name} 的新布尔标志,除非已经定义。此命令是健壮的 (robust)。

  \cmditem{toggletrue}{name}

  将布尔标志 \prm{name} 设置为 \texttt{true}。此命令是健壮的 (robust),可以使用 \cs{global} 修饰。如果标志未定义,
  它将发出错误。

  \cmditem{togglefalse}{name}

  将布尔标志 \prm{name} 设置为 \texttt{false}。此命令是健壮的 (robust),可以使用 \cs{global} 修饰。如果标志未定义,
  它将发出错误。

  \cmditem{settoggle}{name}{value}

  将布尔标志 \prm{name} 设置为 \prm{value},可以是 \texttt{true} 或 \texttt{false}。此命令是健壮的 (robust),可以
  使用 \cs{global} 修饰。如果标志未定义,它将发出错误。

  \cmditem{iftoggle}{name}{true}{false}

  如果布尔标志 \prm{name} 的状态为 \texttt{true},则展开为 \prm{true},否则为 \prm{false}。如果标志未定义,
  此命令将发出错误。

  \cmditem{nottoggle}{name}{not true}{not false}

  类似于 \cmd{iftoggle},但对测试进行否定。

\end{ltxsyntax}

\subsection{通用测试}
\label{aut:tst}

\subsubsection{宏测试}
\label{aut:tst:def}

\begin{ltxsyntax}

  \cmditem{ifdef}{control sequence}{true}{false}

  如果 \prm{control sequence} 已定义,则展开为 \prm{true},否则为 \prm{false}。请注意,即使控制序列的含义是
  \cmd{relax},它也将被视为已定义。此命令是对 \etex 原语 \cmd{ifdefined} 的 \LaTeX 包装。

  \cmditem{ifcsdef}{csname}{true}{false}

  类似于 \cmd{ifdef},但它以控制序列名称作为其第一个参数。此命令是对 \etex 原语 \cmd{ifcsname} 的 \LaTeX 包装。

  \cmditem{ifundef}{control sequence}{true}{false}

  如果 \prm{control sequence} 未定义,则展开为 \prm{true},否则为 \prm{false}。除了反转测试的逻辑外,此命令还
  不同于 \cmd{ifdef},它还包括如果命令的含义是 \cmd{relax} 则将其视为未定义。

  \cmditem{ifcsundef}{csname}{true}{false}

  类似于 \cmd{ifundef},但它以控制序列名称作为其第一个参数。此命令可用作 \LaTeX 核心中的 \cmd{@ifundefined} 测试
  的替代品。

  \cmditem{ifdefmacro}{control sequence}{true}{false}

  如果 \prm{control sequence} 已定义且是一个宏,则展开为 \prm{true},否则为 \prm{false}。

  \cmditem{ifcsmacro}{csname}{true}{false}

  类似于 \cmd{ifdefmacro},但它以控制序列名称作为其第一个参数。

  \cmditem{ifdefparam}{control sequence}{true}{false}

  如果 \prm{control sequence} 已定义且是具有一个或多个参数的宏,则展开为 \prm{true},否则为 \prm{false}。

  \cmditem{ifcsparam}{csname}{true}{false}

  类似于 \cmd{ifdefparam},但它以控制序列名称作为其第一个参数。

  \cmditem{ifdefprefix}{control sequence}{true}{false}

  如果 \prm{control sequence} 已定义且是以 \cs{long} 和/或 \cs{protected} 为前缀的宏,则展开为 \prm{true},
  否则为 \prm{false}。请注意,\cs{outer} 宏无法进行测试。

  \cmditem{ifcsprefix}{csname}{true}{false}

  类似于 \cmd{ifdefprefix},但它以控制序列名称作为其第一个参数。

  \cmditem{ifdefprotected}{control sequence}{true}{false}

  如果 \prm{control sequence} 已定义且是以 \cs{protected} 为前缀的宏,则展开为 \prm{true},否则为 \prm{false}。

  \cmditem{ifcsprotected}{csname}{true}{false}

  类似于 \cmd{ifdefprotected},但它以控制序列名称作为其第一个参数。

  \cmditem{ifdefltxprotect}{control sequence}{true}{false}

  如果 \prm{control sequence} 已定义且是一个 \LaTeX 保护外壳,则执行 \prm{true},否则为 \prm{false}。此命令是
  健壮的 (robust)。它将检测通过 \cmd{DeclareRobustCommand} 或类似技术定义的命令。

  \cmditem{ifcsltxprotect}{csname}{true}{false}

  类似于 \cmd{ifdefltxprotect},但它以控制序列名称作为其第一个参数。

  \cmditem{ifdefempty}{control sequence}{true}{false}

  如果 \prm{control sequence} 已定义且是一个没有参数且替换文本为空的宏,则展开为 \prm{true},否则为 \prm{false}。
  与 \cmd{ifx} 不同,此测试忽略了 \prm{command} 的前缀。

  \cmditem{ifcsempty}{csname}{true}{false}

  类似于 \cmd{ifdefempty},但它以控制序列名称作为其第一个参数。

  \cmditem{ifdefvoid}{control sequence}{true}{false}

  如果 \prm{control sequence} 未定义,或其含义是 \cmd{relax},或是一个没有参数且替换文本为空的控制序列,则展开为
  \prm{true},否则为 \prm{false}。

  \cmditem{ifcsvoid}{csname}{true}{false}

  类似于 \cmd{ifdefvoid},但它以控制序列名称作为其第一个参数。

  \cmditem{ifdefequal}{control sequence}{control sequence}{true}{false}

  比较两个控制序列,如果它们在 \cmd{ifx} 的意义上相等,则展开为 \prm{true},否则为 \prm{false}。与 \cmd{ifx} 不同,
  此测试还将两个控制序列都未定义或含义为 \cmd{relax} 视为 \prm{false}。

  \cmditem{ifcsequal}{csname}{csname}{true}{false}

  类似于 \cmd{ifdefequal},但它以控制序列名称作为其参数。

  \cmditem{ifdefstring}{command}{string}{true}{false}

  将 \prm{command} 的替换文本与 \prm{string} 进行比较,如果它们相等,则执行 \prm{true},否则执行 \prm{false}。
  在测试中,\prm{command} 和 \prm{string} 都不会被展开,并且比较是不受类别码影响的。在 \prm{string} 参数中的控制
  序列标记将被解标记并视为字符串。此命令是健壮的 (robust)。请注意,它只考虑 \prm{command} 的替换文本。例如,这个测试

  \begin{ltxcode}
    \long\edef\mymacro#1#2{\string&}
    \ifdefstring{\mymacro}{&}{true}{false}
  \end{ltxcode}
  %
  将得到 \prm{true}。忽略了 \cmd{mymacro} 的前缀和参数,以及替换文本中的类别码。

  \cmditem{ifcsstring}{csname}{string}{true}{false}

  类似于 \cmd{ifdefstring},但它以控制序列名称作为其第一个参数。

  \cmditem{ifdefstrequal}{command}{command}{true}{false}

  执行两个命令的不受类别码影响的字符串比较。此命令类似于 \cmd{ifdefstring},但要比较的两个参数都是宏。此命令是健壮的 (robust)。

  \cmditem{ifcsstrequal}{csname}{csname}{true}{false}

  类似于 \cmd{ifdefstrequal},但它以控制序列名称作为其参数。

\end{ltxsyntax}

\subsubsection{计数器和长度测试}
\label{aut:tst:cnt}

\begin{ltxsyntax}

  \cmditem{ifdefcounter}{control sequence}{true}{false}

  如果\prm{control sequence}是由\cmd{newcount}分配的 \TeX \cmd{count} 寄存器,则扩展为\prm{true},
  否则为\prm{false}。

  \cmditem{ifcscounter}{csname}{true}{false}

  类似于\cmd{ifdefcounter},但它以控制序列名称作为其第一个参数。

  \cmditem{ifltxcounter}{name}{true}{false}

  如果\prm{name}是由\cmd{newcounter}分配的 \LaTeX 计数器,则扩展为\prm{true},否则为\prm{false}。

  \cmditem{ifdeflength}{control sequence}{true}{false}

  如果\prm{control sequence}是由\cmd{newskip}或\cmd{newlength}分配的 \TeX \cmd{skip} 寄存器,则扩展为
  \prm{true},否则为\prm{false}。

  \cmditem{ifcslength}{csname}{true}{false}

  类似于\cmd{ifdeflength},但它以控制序列名称作为其第一个参数。

  \cmditem{ifdefdimen}{control sequence}{true}{false}

  如果\prm{control sequence}是由\cmd{newdimen}分配的 \TeX \cmd{dimen} 寄存器,则扩展为\prm{true},否则
  为\prm{false}。

  \cmditem{ifcsdimen}{csname}{true}{false}

  类似于\cmd{ifdefdimen},但它以控制序列名称作为其第一个参数。

\end{ltxsyntax}

\subsubsection{字符串测试}
\label{aut:tst:str}

\begin{ltxsyntax}

  \cmditem{ifstrequal}{string}{string}{true}{false}

  比较两个字符串,如果它们相等则执行\prm{true},否则执行\prm{false}。在测试中不会展开字符串,
  比较时不考虑类别码。任何\prm{string}参数中的控制序列标记都将被解记号化并视为字符串。此命令是健壮的 (robust)。

  \cmditem{ifstrempty}{string}{true}{false}

  如果\prm{string}为空,则扩展为\prm{true},否则扩展为\prm{false}。在测试中不会展开\prm{string}。

  \cmditem{ifblank}{string}{true}{false}

  如果\prm{string}为空白(即空或只包含空格),则扩展为\prm{true},否则扩展为\prm{false}。在测试中不会展开
  \prm{string}。

  \cmditem{notblank}{string}{not true}{not false}

  类似于\cmd{ifblank},但否定了测试。

\end{ltxsyntax}

\subsubsection{算术测试}
\label{aut:tst:num}

\begin{ltxsyntax}

  \cmditem{ifnumcomp}{integer expression}{relation}{integer expression}{true}{false}

  根据 \prm{relation} 比较两个整数表达式,根据结果展开为 \prm{true} 或 \prm{false}。 \prm{relation} 可以是 |<|、|>| 或 |=|。两个整数表达式将使用 \cmd{numexpr} 处理。一个 \prm{integer expression} 可以是在此上下文中有效的任意代码。所有算术表达式都可以包含空格。以下是一些例子:

  \begin{ltxcode}
    \ifnumcomp{<<3>>}{<<>>>}{<<6>>}{true}{<<false>>}
    \ifnumcomp{<<(7 + 5) / 2>>}{<<=>>}{<<6>>}{<<true>>}{false}
    \ifnumcomp{<<(7+5) / 4>>}{<<>>>}{<<3*(12-10)>>}{true}{<<false>>}
    \newcounter{countA}
    \setcounter{countA}{<<6>>}
    \newcounter{countB}
    \setcounter{countB}{<<5>>}
    \ifnumcomp{<<\value{countA} * \value{countB}/2}>>{<<=>>}{<<15>>}{<<true>>}{false}
    \ifnumcomp{<<6/2>>}{<<=>>}{<<5/2>>}{<<true>>}{false}
  \end{ltxcode}
  %
  从技术上讲,此命令是 \LaTeX 对 \TeX 原语 \cmd{ifnum} 的包装,包含了 \cmd{numexpr}。注意,\cmd{numexpr} 将对所有整数表达式的结果进行四舍五入,即在比较之前,两个表达式都将被处理并四舍五入。在上述示例的最后一行中,第二个表达式的结果是2.5,四舍五入为3,因此 \cmd{ifnumcomp} 将展开为 \prm{true}。

  \cmditem{ifnumequal}{integer expression}{integer expression}{true}{false}

  |\ifnumcomp{...}{=}{...}{...}{...}| 的替代语法。

  \cmditem{ifnumgreater}{integer expression}{integer expression}{true}{false}

  |\ifnumcomp{...}{>}{...}{...}{...}| 的替代语法。

  \cmditem{ifnumless}{integer expression}{integer expression}{true}{false}

  |\ifnumcomp{...}{<}{...}{...}{...}| 的替代语法。

  \cmditem{ifnumodd}{integer expression}{true}{false}

  对整数表达式进行评估,如果结果为奇数则展开为 \prm{true},否则展开为 \prm{false}。从技术上讲,此命令是 \LaTeX 对 \TeX 原语 \cmd{ifodd} 的包装,包含了 \cmd{numexpr}。

  \cmditem{ifdimcomp}{dimen expression}{relation}{dimen expression}{true}{false}

  根据 \prm{relation} 比较两个尺寸表达式,并根据结果展开为 \prm{true} 或 \prm{false}。 \prm{relation} 可以是 |<|、|>| 或 |=|。两个尺寸表达式都将使用 \cmd{dimexpr} 处理。一个 \prm{dimen expression} 可以是在此上下文中有效的任意代码。所有算术表达式都可以包含空格。以下是一些例子:

  \begin{ltxcode}
    \ifdimcomp{<<1cm>>}{<<=>>}{<<28.45274pt>>}{<<true>>}{false}
    \ifdimcomp{<<(7pt + 5pt) / 2>>}{<<<>>}{2pt}{true}{<<false>>}
    \ifdimcomp{<<(3.725pt + 0.025pt) * 2>>}{<<<>>}{<<7pt>>}{true}{<<false>>}
    \newlength{\lengthA}
    \setlength{\lengthA}{<<7.25pt>>}
    \newlength{\lengthB}
    \setlength{\lengthB}{<<4.75pt>>}
    \ifdimcomp{<<(\lengthA + \lengthB) / 2>>}{<<>>>}{<<2.75pt * 2>>}{<<true>>}{false}
    \ifdimcomp{<<(\lengthA + \lengthB) / 2>>}{<<>>>}{<<25pt / 6>>}{<<true>>}{false}
  \end{ltxcode}
  %
  从技术上讲,此命令是 \LaTeX 对 \TeX 原语 \cmd{ifdim} 的包装,包含了 \cmd{dimexpr}。由于 \cmd{ifdimcomp} 和 \cmd{ifnumcomp} 都是可展开的,它们也可以嵌套使用:

  \begin{ltxcode}
    <<\ifnumcomp>>{<<\ifdimcomp>>{<<5pt+5pt>>}{<<=>>}{<<10pt>>}{<<1>>}{<<0>>}}{<<>>>}{<<0>>}{<<true>>}{false}
  \end{ltxcode}

  \cmditem{ifdimequal}{dimen expression}{dimen expression}{true}{false}

  |\ifdimcomp{...}{=}{...}{...}{...}| 的替代语法。

  \cmditem{ifdimgreater}{dimen expression}{dimen expression}{true}{false}

  |\ifdimcomp{...}{>}{...}{...}{...}| 的替代语法。

  \cmditem{ifdimless}{dimen expression}{dimen expression}{true}{false}

  |\ifdimcomp{...}{<}{...}{...}{...}| 的替代语法。

\end{ltxsyntax}

\subsubsection{布尔表达式}
\label{aut:tst:bol}

本节中的命令是 \sty{ifthen} 包提供的 \cmd{ifthenelse} 命令的替代品。它们具有相同的目的,但在语法、概念和
实现上有所不同。与 \cmd{ifthenelse} 不同,它们本身不提供任何测试,而是作为其他测试的前端。任何满足特定语法
要求的测试都可以用于布尔表达式。

\begin{ltxsyntax}

  \cmditem{ifboolexpr}{expression}{true}{false}

  对 \prm{expression} 进行评估,如果为真则执行 \prm{true},否则执行 \prm{false}。从左到右顺序评估
  \prm{expression}。以下元素在 \prm{expression} 中可用,并在下面更详细地讨论:测试运算符 \texttt{togl}、
  \texttt{bool}、\texttt{test};逻辑运算符 \texttt{not}、\texttt{and}、\texttt{or};子表达式定界符
  \texttt{(...)}。可以自由使用空格、制表符和换行符来可视化排列 \prm{expression}。在 \prm{expression}
  中不允许空白行。此命令是健壮的 (robust)。

  \cmditem{ifboolexpe}{expression}{true}{false}

  \cmd{ifboolexpr} 的可展开版本,在仅限展开的上下文中(例如 \cmd{edef} 或 \cmd{write} 操作)可以处理。
  请注意,\prm{expression} 中使用的所有测试必须是可展开的,即使 \cmd{ifboolexpe} 不位于仅展开的上下文中。

  \cmditem{whileboolexpr}{expression}{code}

  类似于 \cmd{ifboolexpr} 对 \prm{expression} 进行评估,并在表达式为真时重复执行 \prm{code}。\prm{code}
  可以是任何有效的 \TeX 或 \LaTeX 代码。此命令是健壮的 (robust)。

  \cmditem{unlessboolexpr}{expression}{code}

  类似于 \cmd{whileboolexpr},但否定了 \prm{expression},即只有在表达式为真时才会重复执行 \prm{code}。
  此命令是健壮的 (robust)。

\end{ltxsyntax}
%
以下测试运算符可用于 \prm{expression}:

\begin{marglist}
  \appto\marglistfont{\verbatimfont}

  \item[togl]

  使用 \texttt{togl} 运算符测试使用 \cmd{newtoggle} 定义的标志的状态。例如:

  \begin{ltxcode}
    <<\iftoggle{mytoggle}>>{true}{false}
  \end{ltxcode}
  %
  变为

  \begin{ltxcode}
    \ifboolexpr{ <<togl>> {<<mytoggle>>} }{true}{false}
  \end{ltxcode}
  %
  \texttt{togl} 运算符可以与 \cmd{ifboolexpr} 和 \cmd{ifboolexpe} 都一起使用。

  \item[bool]

  使用 \texttt{bool} 运算符基于纯 \TeX 语法执行布尔测试,即通常使用如下方式的任何测试:

  \begin{ltxcode}
    <<\iftest>> true<<\else>> false<<\fi>>
  \end{ltxcode}
  %
  这包括所有使用 \cmd{newif} 定义的标志以及诸如 \cmd{ifmmode} 等 \TeX 原语。在表达式中使用标志或
  原语时,忽略 \cmd{if} 前缀。例如:

  \begin{ltxcode}
    <<\ifmmode>> true\else false\fi
    <<\ifmytest>> true\else false\fi
  \end{ltxcode}
  %
  变为

  \begin{ltxcode}
    \ifboolexpr{ <<bool>> {<<mmode>>} }{true}{false}
    \ifboolexpr{ <<bool>> {<<mytest>>} }{true}{false}
  \end{ltxcode}
  %
  这也适用于使用 \cmd{newbool} 定义的标志(见 \secref{aut:bo1:bol})。在这种情况下,

  \begin{ltxcode}
    <<\ifbool{mybool}>>{true}{false}
  \end{ltxcode}
  %
  变为

  \begin{ltxcode}
    \ifboolexpr{ <<bool>> {<<mybool>>} }{true}{false}
  \end{ltxcode}
  %
  \texttt{bool} 运算符可以与 \cmd{ifboolexpr} 和 \cmd{ifboolexpe} 都一起使用。

  \item[test]

  使用 \texttt{test} 运算符基于 \LaTeX 语法执行测试,即通常使用如下方式的任何测试:

  \begin{ltxcode}
    <<\iftest>>{<<true>>}{<<false>>}
  \end{ltxcode}
  %
  这适用于基于 \LaTeX 语法的所有宏,即宏必须接受 \prm{true} 和 \prm{false} 参数,并且这些参数必须
  是最终参数。例如:

  \begin{ltxcode}
    <<\ifdef>>{\somemacro}<<{true}{false}>>
    <<\ifdimless>>{\textwidth}{365pt}<<{true}{false}>>
      <<\ifnumcomp>>{\value{somecounter}}{>}{3}<<{true}{false}>>
  \end{ltxcode}

  在 \prm{expression} 中使用这些测试时,它们的 \prm{true} 和 \prm{false} 参数被省略。例如:

  \begin{ltxcode}
    <<\ifcsdef{mymacro}>>{true}{false}
  \end{ltxcode}
  %
  变为

  \begin{ltxcode}
    \ifboolexpr{ <<test>> {<<\ifcsdef{mymacro}>>} }{true}{false}
  \end{ltxcode}
  %
  和

  \begin{ltxcode}
    <<\ifnumcomp{\value{mycounter}}{>}{3}>>{true}{false}
  \end{ltxcode}
  %
  变为

  \begin{ltxcode}
    \ifboolexpr{
      <<test>> {<<\ifnumcomp{\value{mycounter}}{>}{3}>>}
    }
    {true}
    {false}
  \end{ltxcode}
  %
  \texttt{test} 运算符可以无限制地与 \cmd{ifboolexpr} 一起使用。它也可以与 \cmd{ifboolexpe} 一起
  使用,前提是测试是可展开的。一些在 \secref{aut:tst} 中的通用测试是健壮的 (robust)。,即使 \cmd{ifboolexpe}
  不位于仅展开的上下文中,也不能与之一起使用。如果测试不可展开,请使用 \cmd{ifboolexpr}。
\end{marglist}

由于 \cmd{ifboolexpr} 和 \cmd{ifboolexpe} 需要处理开销,通常在单个测试中使用它们没有好处。 \secref{aut:tst} 中的独立测试比 \texttt{test} 更有效率,\secref{aut:bo1:bol} 中的 \cmd{ifbool} 比 \texttt{bool} 更有效率,\secref{aut:bo1:tgl} 中的 \cmd{iftoggle} 比 \texttt{togl} 更有效率。 \cmd{ifboolexpr} 和 \cmd{ifboolexpe} 的目的在于它们支持逻辑运算符和子表达式。以下逻辑运算符可用于 \prm{expression}:

\begin{marglist}
  \appto\marglistfont{\verbatimfont}

  \item[not]

  \texttt{not} 运算符否定紧接其后的元素的真值。您可以在 \texttt{togl}、\texttt{bool}、\texttt{test} 和子表达式前加上 \texttt{not}。例如:

  \begin{ltxcode}
    \ifboolexpr{
      <<not>> bool {mybool}
    }
    {true}
    {false}
  \end{ltxcode}
  %
  如果 \texttt{mybool} 为 false,则结果为 \prm{true};如果 \texttt{mybool} 为 true,则结果为 \prm{false}。并且

  \begin{ltxcode}
    \ifboolexpr{
      <<not (>> bool {boolA} or bool {boolB} <<)>>
    }
    {true}
    {false}
  \end{ltxcode}
  %
  如果 \texttt{boolA} 和 \texttt{boolB} 都为 false,则结果为 \prm{true}。

  \item[and]

  \texttt{and} 运算符表示合取(\emph{a} 和 \emph{b} 都为真)。如果 \texttt{and} 连接的所有元素都为真,则 \prm{expression} 为真。例如:

  \begin{ltxcode}
    \ifboolexpr{
      bool {boolA} <<and>> bool {boolB}
    }
    {true}
    {false}
  \end{ltxcode}
  %
  如果两个 \texttt{bool} 测试都为真,则结果为 \prm{true}。不提供 \texttt{nand} 运算符(否定的合取,即不是两者都为真),但可以通过在否定子表达式中使用 \texttt{and} 来表达。例如:

  \begin{ltxcode}
    bool {boolA} <<nand>> bool {boolB}
  \end{ltxcode}
  %
  可以写为

  \begin{ltxcode}
    <<not>> <<(>> bool {boolA} <<and>> bool {boolB} <<)>>
  \end{ltxcode}

  \item[or]

  \texttt{or} 运算符表示非排他的析取(\emph{a} 或 \emph{b} 或两者都为真)。如果至少有一个使用 \texttt{or} 连接的元素为真,则 \prm{expression} 为真。例如:

  \begin{ltxcode}
    \ifboolexpr{
      togl {toglA} <<or>> togl {toglB}
    }
    {true}
    {false}
  \end{ltxcode}
  %
  如果至少一个或两个测试为真,则结果为 \prm{true}。不提供 \texttt{nor} 运算符(否定的非排他的析取,即两者都不为真),但可以通过在否定子表达式中使用 \texttt{or} 来表达。例如:

  \begin{ltxcode}
    bool {boolA} <<nor>> bool {boolB}
  \end{ltxcode}
  %
  可以写为

  \begin{ltxcode}
    <<not>> <<(>> bool {boolA} <<or>> bool {boolB} <<)>>
  \end{ltxcode}

  \item[(...)]

  括号用于限定 \prm{expression} 中的子表达式。子表达式会被计算,其计算结果会被视为封闭表达式中的单个真值。子表达式可以嵌套。例如,表达式:

  \begin{ltxcode}
    <<(>> bool {boolA} or bool {boolB} <<)>>
    and
    <<(>> bool {boolC} or bool {boolD} <<)>>
  \end{ltxcode}
  %
  如果两个子表达式都为真,则结果为真;即如果 \texttt{boolA}/\texttt{boolB} 中至少一个为真,且
  \texttt{boolC}/\texttt{boolD} 中至少一个为真,则结果为真。如果所有元素都使用 \texttt{and} 或 \texttt{or}
  连接,则通常不需要子表达式。例如,表达式

  \begin{ltxcode}
    bool {boolA} <<and>> bool {boolB} <<and>> {boolC} <<and>> bool {boolD}
    bool {boolA} <<or>> bool {boolB} <<or>> {boolC} <<or>> bool {boolD}
  \end{ltxcode}
  %
  将产生预期的结果:第一个表达式如果所有元素都为真,则结果为真;第二个表达式如果至少一个元素为真,则结果为真。
  但是,当组合 \texttt{and} 和 \texttt{or} 时,建议始终将元素分组到子表达式中,以避免可能因形式布尔表达式的
  语义与自然语言的语义之间的差异而产生的误解。例如,以下表达式

  \begin{ltxcode}
    bool {<<coffee>>} <<and>> bool {<<milk>>} <<or>> bool {<<sugar>>}
  \end{ltxcode}
  %
  如果 \texttt{sugar} 为真,则始终为真,因为 \texttt{or} 运算符将 \texttt{and} 评估结果作为输入。与英语中的
  表达含义相反,它不是按照以下方式处理:

  \begin{ltxcode}
    bool {<<coffee>>} <<and>> <<(>> bool {<<milk>>} <<or>> bool {<<sugar>>} <<)>>
  \end{ltxcode}
  %
  但是严格地从左到右进行评估:

  \begin{ltxcode}
    <<(>> bool {<<coffee>>} <<and>> bool {<<milk>>} <<)>> <<or>> bool {<<sugar>>}
  \end{ltxcode}
  %
  这可能不是您想要的顺序。

\end{marglist}

\subsection{列表处理}
\label{aut:lst}

\subsubsection{用户输入}
\label{aut:lst:inp}

本节工具主要用于处理用户输入。在构建供包内部使用的列表时,可能更倾向于使用第 \ref{aut:lst:int} 节中的工具,
因为它们允许测试元素是否在列表中。

\begin{ltxsyntax}

  \cmditem{DeclareListParser}{command}{separator}

  此命令定义一个列表解析器,类似于下面的 \cmd{docsvlist} 命令,其定义如下:

  \begin{ltxcode}
    \DeclareListParser{\docsvlist}{,}
  \end{ltxcode}
  %
  请注意,列表解析器对 \prm{separator} 的分类代码很敏感。

  \cmditem*{DeclareListParser*}{command}{separator}

  \cmd{DeclareListParser} 的星号变体定义了类似于下面的 \cmd{forcsvlist} 命令的列表解析器,其定义如下:

  \begin{ltxcode}
    \DeclareListParser*{\forcsvlist}{,}
  \end{ltxcode}

  \cmditem{docsvlist}{item, item, ...}

  此命令循环处理逗号分隔的项目列表,并对列表中的每个项目执行辅助命令 \cmd{do},将项目作为参数传递。与
  \LaTeX 核心中的 \cmd{@for} 循环不同,\cmd{docsvlist} 是可展开的。通过适当定义 \cmd{do},列表可以在
  \cmd{edef} 或可比较的上下文中处理。您可以在 \cmd{do} 的替换文本末尾使用 \cmd{listbreak} 来停止处理
  并丢弃列表中的剩余项目。列表分隔符后的空格将被忽略。如果项目包含逗号或以空格开头,则必须用花括号包装。
  在处理列表时,花括号将被移除。以下是使用示例,将逗号分隔的列表打印为 \env{itemize} 环境:

  \begin{ltxcode}
    \begin{itemize}
      \renewcommand*{\do}[1]{\item #1}
            \docsvlist{item1, item2, {item3a, item3b}, item4}
    \end{itemize}
  \end{ltxcode}
  %
  还有另一个示例:

  \begin{ltxcode}
    \renewcommand*{\do}[1]{* #1\MessageBreak}
    \PackageInfo{mypackage}{%
      Example list:\MessageBreak
      \docsvlist{item1, item2, {item3a, item3b}, item4}}
  \end{ltxcode}
  %
  在此示例中,列表将作为信息消息写入日志文件。列表处理发生在 \cmd{write} 操作期间。

  \cmditem{forcsvlist}{handler}{item, item, ...}

  此命令与 \cmd{docsvlist} 类似,但 \cmd{do} 被在调用时指定的 \prm{handler} 替换。 \prm{handler}
  也可以是一系列命令,只要最后给出的命令将项目作为尾随参数。例如,以下代码将把逗号分隔的项目列表转换为名为
  \cmd{mylist} 的内部列表:

  \begin{ltxcode}
    \forcsvlist{\listadd\mylist}{item1, item2, item3}
  \end{ltxcode}

\end{ltxsyntax}

\subsubsection{内部列表}
\label{aut:lst:int}

本节中的工具用于处理内部数据列表。在这个上下文中,<内部列表> 是指一个没有任何参数和前缀的普通宏,
用于收集数据。这些列表使用特殊字符作为内部列表分隔符。\footnote{这个特殊字符是具有类别码 3 的
  \texttt{\string|}。请注意,你不能通过 \cmd{listname} 来排版列表。使用 \cmd{show} 来检查列表。}
在处理用户以列表格式输入的内容时,请参考第 \ref{aut:lst:inp} 节中的工具。

\begin{ltxsyntax}

  \cmditem{listadd}{listmacro}{item}

  这个命令将一个 \prm{item} 添加到 \prm{listmacro} 中。空的 \prm{item} 不会添加到列表中。

  \cmditem{listgadd}{listmacro}{item}

  类似于 \cmd{listadd},但是执行的是全局赋值。

  \cmditem{listeadd}{listmacro}{item}

  类似于 \cmd{listadd},但是 \prm{item} 在定义时被展开。只有新的 \prm{item} 会被展开,
  \prm{listmacro} 不会被展开。如果展开后的 \prm{item} 是空的,那么它不会被添加到列表中。

  \cmditem{listxadd}{listmacro}{item}

  类似于 \cmd{listeadd},但是执行的是全局赋值。

  \cmditem{listcsadd}{listcsname}{item}

  类似于 \cmd{listadd},但是第一个参数是控制序列的名称。

  \cmditem{listcsgadd}{listcsname}{item}

  类似于 \cmd{listcsadd},但是执行的是全局赋值。

  \cmditem{listcseadd}{listcsname}{item}

  类似于 \cmd{listeadd},但是第一个参数是控制序列的名称。

  \cmditem{listcsxadd}{listcsname}{item}

  类似于 \cmd{listcseadd},但是执行的是全局赋值。

  \cmditem{listremove}{listmacro}{item}

  这个命令从 \prm{listmacro} 中移除一个 \prm{item}。空的 \prm{item} 会被忽略。

  \cmditem{listgremove}{listmacro}{item}

  类似于 \cmd{listremove},但是执行的是全局赋值。

  \cmditem{listcsremove}{listcsname}{item}

  类似于 \cmd{listremove},但是第一个参数是控制序列的名称。

  \cmditem{listcsgremove}{listcsname}{item}

  类似于 \cmd{listcsremove},但是执行的是全局赋值。

  \cmditem{dolistloop}{listmacro}

  这个命令遍历 \prm{listmacro} 中的所有项目,并为列表中的每个项目执行辅助命令 \cmd{do},将项目作为
  参数传递。列表循环本身是可展开的。你可以在 \cmd{do} 的替换文本末尾使用 \cmd{listbreak} 来停止处
  理,并丢弃列表中剩余的项目。以下是一个用法示例,将名为 \cmd{mylist} 的内部列表打印为 \env{itemize}
  环境:

  \begin{ltxcode}
    \begin{itemize}
      \renewcommand*{\do}[1]{\item #1}
            \dolistloop{\mylist}
    \end{itemize}
  \end{ltxcode}

  \cmditem{dolistcsloop}{listcsname}

  类似于 \cmd{dolistloop},但是第一个参数是控制序列的名称。

  \cmditem{forlistloop}{handler}{listmacro}

  这个命令类似于 \cmd{dolistloop},但是在调用时,\cmd{do} 被指定的 \prm{handler} 替代。
  \prm{handler} 也可以是一系列命令,只要最后给出的命令将项目作为尾随参数。例如,以下代码将在内部
  列表 \cmd{mylist} 中的所有项目前加上 \cmd{item},在处理列表时计数项目,并在末尾附加项目计数:

  \begin{ltxcode}
    \newcounter{itemcount}
    \begin{itemize}
      \forlistloop{\stepcounter{itemcount}\item}{\mylist}
      \item Total: \number\value{itemcount} items
    \end{itemize}
  \end{ltxcode}

  \cmditem{forlistcsloop}{handler}{listcsname}

  类似于 \cmd{forlistloop},但是第二个参数是控制序列的名称。

  \cmditem{ifinlist}{item}{listmacro}{true}{false}

  如果 \prm{item} 包含在 \prm{listmacro} 中,则执行 \prm{true},否则执行 \prm{false}。请注意,
  此测试使用基于 \TeX{} 的参数扫描器进行模式匹配,以检查搜索字符串是否包含在列表中。这意味着它通常
  比遍历列表中的所有项目更快,但也意味着项目不能包含花括号,否则会有效地隐藏它们。换句话说,在处理
  纯字符串列表而不是可打印数据时,这个宏最有用。在处理可打印文本时,最好使用 \cmd{dolistloop} 来
  检查一个项目是否在列表中,如下所示:

  \begin{ltxcode}
    \renewcommand*{\do}[1]{%
      \ifstrequal{#1}{<<item>>}
      {item found!\listbreak}
      {}}
    \dolistloop{\mylist}
  \end{ltxcode}

  \cmditem{xifinlist}{item}{listmacro}{true}{false}

  类似于 \cmd{ifinlist},但是在测试之前展开了 \prm{item}。

  \cmditem{ifinlistcs}{item}{listcsname}{true}{false}

  类似于 \cmd{ifinlist},但是第二个参数是控制序列的名称。

  \cmditem{xifinlistcs}{item}{listcsname}{true}{false}

  类似于 \cmd{xifinlist},但是第二个参数是控制序列的名称。

\end{ltxsyntax}

\subsection{杂项工具}
\label{aut:msc}

\begin{ltxsyntax}

  \cmditem{rmntonum}{numeral}

  \TeX 原语 \cmd{romannumeral} 可以将整数转换为罗马数字,但是 \TeX 或 \LaTeX 没有提供相反的命令。
  \cmd{rmntonum} 填补了这个空白。它以罗马数字作为参数,并将其转换为相应的整数。由于它是可展开的,
  因此也可以用于计数器赋值或算术测试:

  \begin{ltxcode}
    <<\rmntonum>>{<<mcmxcv>>}
    \setcounter{counter}{<<\rmntonum>>{<<CXVI>>}}
    \ifnumless{<<\rmntonum>>{<<mcmxcviii>>}}{2000}{true}{false}
  \end{ltxcode}
  %
  \prm{numeral} 参数必须是一个文字字符串。在解析之前,它将被去除记号化。罗马数字的解析不区分大小写,
  并忽略参数中的空格。如果参数中存在无效的记号,\cmd{rmntonum} 将展开为 \texttt{-1};空参数将产生
  空字符串。请注意,\cmd{rmntonum} 不会检查罗马数字的形式有效性。例如,\texttt{V} 和 \texttt{VX}
  都会产生 \texttt{5},\texttt{IC} 会产生 \texttt{99},等等。

  \cmditem{ifrmnum}{string}{true}{false}

  如果 \prm{string} 是一个罗马数字,则展开为 \prm{true},否则展开为 \prm{false}。在执行测试之前,
  \prm{string} 将被去除记号化。这个测试不区分大小写,并忽略 \prm{string} 中的空格。请注意,
  \cmd{ifrmnum} 不会检查罗马数字的形式有效性。例如,\texttt{V} 和 \texttt{VXV} 都会展开为
  \prm{true}。严格来说,\cmd{ifrmnum} 的作用是解析 \prm{string} 以确定它是否由可以构成有效罗马数字
  的字符组成,但它不会检查它们是否真的是有效的罗马数字。

\end{ltxsyntax}

\section{报告问题}

\sty{etoolbox} 宏包的开发代码托管在 GitHub 上: \url{https://github.com/josephwright/etoolbox} ,这是
记录该包任何问题的最佳位置。

\section{修订历史}

本修订历史记录了与此软件包用户相关的变更列表。不包括更多技术性质的更改,这些更改不影响用户界面或软件包的行为。
如果修订历史中的条目说明某个功能已经“改进”或“扩展”,这表示是一种语法向后兼容的修改,比如向现有命令添加可选参数。
如果修订历史中的条目说明某个功能已被“修改”,那需要注意,这表示一种可能需要在一些,希望是少数情况下,修改现有
文档的修改。右侧的数字表示本手册的相关部分。

\begin{changelog}

  \begin{release}{2.5k}{2020-10-05}
    \item 内部更新
  \end{release}

  \begin{release}{2.5j}{2020-08-24}
    \item 跟踪 \LaTeXe{} 核心变更
  \end{release}

  \begin{release}{2.5i}{2020-07-13}
    \item 跟踪 \LaTeXe{} 核心变更
  \end{release}

  \begin{release}{2.5h}{2019-09-21}
    \item 添加缺失的 \cmd{gundef}
  \end{release}

  \begin{release}{2.5g}{2019-09-09}
    \item 更新对 \LaTeX{} 核心变更前 \cmd{begin} 和 \cmd{end} 的修补
  \end{release}

  \begin{release}{2.5f}{2018-08-18}
    \item 修复 \cmd{ifdefempty}、\cmd{ifcsempty}、\cs{ifdefvoid} 和 \cmd{ifcsvoid} 用于扩展为空格符的宏时出现的问题
  \end{release}

  \begin{release}{2.5e}{2018-02-11}
    \item 在 \cmd{DeclareListParser} 中对空列表分隔符进行更多工作
  \end{release}

  \begin{release}{2.5d}{2018-02-10}
    \item 允许在 \cmd{DeclareListParser} 中使用空列表分隔符
  \end{release}

  \begin{release}{2.5c}{2018-02-06}
    \item 修复了由 v2.5b 引入的 \cmd{forcsvlist} 的问题
  \end{release}

  \begin{release}{2.5b}{2018-02-04}
    \item 在某些内部步骤中保留大括号
    \item 对列表处理器进行内部性能改进
  \end{release}

  \begin{release}{2.5a}{2018-02-03}
    \item 对列表处理器进行内部性能改进
  \end{release}

  \begin{release}{2.5}{2017-11-22}
    \item 添加了 \cmd{csgundef}\see{aut:def:def}
    \item 添加了 \cmd{gundef}\see{aut:def:def}
    \item 允许扫描包含换行符的宏
  \end{release}

  \begin{release}{2.4}{2017-01-02}
    \item 将 \cmd{listdel} 重命名为 \cmd{listremove}(名称冲突)\see{aut:lst:int}
    \item 将 \cmd{listgdel} 重命名为 \cmd{listgremove}(名称冲突)\see{aut:lst:int}
  \end{release}

  \begin{release}{2.3}{2016-12-26}
    \item 添加了 \cmd{listdel}\see{aut:lst:int}
    \item 添加了 \cmd{listgdel}\see{aut:lst:int}
  \end{release}

  \begin{release}{2.2b}{2016-12-01}
    \item 修复了对某些类型的 \LaTeX 健壮 (robust) 命令的 \cmd{ifdefltxprotect} 的问题
    \item 在 \cmd{robustify} 处理后删除了冗余的宏
  \end{release}

  \begin{release}{2.2a}{2015-08-02}
    \item 修复了 \cmd{ifblank}/\cmd{notblank} 中的健壮性 (robustness) 错误
  \end{release}

  \begin{release}{2.2}{2015-05-04}
    \item 添加了 \cmd{csmeaning}\see{aut:def:def}
  \end{release}

  \begin{release}{2.1d}{2015-03-19}
    \item 修复了与 \sty{bm} 和某些类相关的问题
  \end{release}

  \begin{release}{2.1c}{2015-03-15}
    \item 修复了 \cmd{ifpatchable} 中的空格问题
    \item 修复了 \cmd{patchcmd} 中的空格问题
    \item 修复了 \cmd{robustify} 中的空格问题
  \end{release}

  \begin{release}{2.1b}{2015-03-10}
    \item 对文档进行了轻微修订
  \end{release}

  \begin{release}{2.1a}{2015-03-10}
    \item 新的维护者:Joseph Wright
    \item 在较新的 \LaTeX 核心发布中跳过加载 \sty{etex} 包
  \end{release}

  \begin{release}{2.1}{2011-01-03}
    \item 添加了 \cmd{AtBeginEnvironment}\see{use:env}
    \item 添加了 \cmd{AtEndEnvironment}\see{use:env}
    \item 添加了 \cmd{BeforeBeginEnvironment}\see{use:env}
    \item 添加了 \cmd{AfterEndEnvironment}\see{use:env}
    \item 添加了 \cmd{ifdefstrequal}\see{aut:tst:def}
    \item 添加了 \cmd{ifcsstrequal}\see{aut:tst:def}
    \item 添加了 \cmd{ifdefcounter}\see{aut:tst:cnt}
    \item 添加了 \cmd{ifcscounter}\see{aut:tst:cnt}
    \item 添加了 \cmd{ifltxcounter}\see{aut:tst:cnt}
    \item 添加了 \cmd{ifdeflength}\see{aut:tst:cnt}
    \item 添加了 \cmd{ifcslength}\see{aut:tst:cnt}
    \item 添加了 \cmd{ifdefdimen}\see{aut:tst:cnt}
    \item 添加了 \cmd{ifcsdimen}\see{aut:tst:cnt}
  \end{release}

  \begin{release}{2.0a}{2010-09-12}
    \item 修复了 \cmd{patchcmd}、\cmd{apptocmd}、\cmd{pretocmd} 中的错误\see{aut:pat}
  \end{release}

  \begin{release}{2.0}{2010-08-21}

    \item 添加了 \cmd{csshow}\see{aut:def:def}
    \item 添加了 \cmd{DeclareListParser*}\see{aut:lst:inp}
    \item 添加了 \cmd{forcsvlist}\see{aut:lst:inp}
    \item 添加了 \cmd{forlistloop}\see{aut:lst:int}
    \item 添加了 \cmd{forlistcsloop}\see{aut:lst:int}
    \item 允许在宏测试中测试 \cmd{par}\see{aut:tst:def}
    \item 修复了一些错误

  \end{release}

  \begin{release}{1.9}{2010-04-10}

    \item 改进了 \cmd{letcs}\see{aut:def:def}
    \item 改进了 \cmd{csletcs}\see{aut:def:def}
    \item 改进了 \cmd{listeadd}\see{aut:lst:int}
    \item 改进了 \cmd{listxadd}\see{aut:lst:int}
    \item 添加了 \cmd{notblank}\see{aut:tst:str}
    \item 添加了 \cmd{ifnumodd}\see{aut:tst:num}
    \item 添加了 \cmd{ifboolexpr}\see{aut:tst:bol}
    \item 添加了 \cmd{ifboolexpe}\see{aut:tst:bol}
    \item 添加了 \cmd{whileboolexpr}\see{aut:tst:bol}
    \item 添加了 \cmd{unlessboolexpr}\see{aut:tst:bol}

  \end{release}

  \begin{release}{1.8}{2009-08-06}

    \item 改进了 \cmd{deflength}\see{use:cal}
    \item 添加了 \cmd{ifnumcomp}\see{aut:tst:num}
    \item 添加了 \cmd{ifnumequal}\see{aut:tst:num}
    \item 添加了 \cmd{ifnumgreater}\see{aut:tst:num}
    \item 添加了 \cmd{ifnumless}\see{aut:tst:num}
    \item 添加了 \cmd{ifdimcomp}\see{aut:tst:num}
    \item 添加了 \cmd{ifdimequal}\see{aut:tst:num}
    \item 添加了 \cmd{ifdimgreater}\see{aut:tst:num}
    \item 添加了 \cmd{ifdimless}\see{aut:tst:num}

  \end{release}

  \begin{release}{1.7}{2008-06-28}
    \item 将 \cmd{AfterBeginDocument} 重命名为 \cmd{AfterEndPreamble}(名称冲突)\see{use:pre}
    \item 解决了与 \sty{hyperref} 的冲突
    \item 稍微重新排列了手册内容
  \end{release}

  \begin{release}{1.6}{2008-06-22}
    \item 改进了 \cmd{robustify}\see{use:pat}
    \item 改进了 \cmd{patchcmd} 和 \cmd{ifpatchable}\see{aut:pat}
    \item 修改并改进了 \cmd{apptocmd}\see{aut:pat}
    \item 修改并改进了 \cmd{pretocmd}\see{aut:pat}
    \item 添加了 \cmd{ifpatchable*}\see{aut:pat}
    \item 添加了 \cmd{tracingpatches}\see{aut:pat}
    \item 添加了 \cmd{AfterBeginDocument}\see{use:pre}
    \item 添加了 \cmd{ifdefmacro}\see{aut:tst:def}
    \item 添加了 \cmd{ifcsmacro}\see{aut:tst:def}
    \item 添加了 \cmd{ifdefprefix}\see{aut:tst:def}
    \item 添加了 \cmd{ifcsprefix}\see{aut:tst:def}
    \item 添加了 \cmd{ifdefparam}\see{aut:tst:def}
    \item 添加了 \cmd{ifcsparam}\see{aut:tst:def}
    \item 添加了 \cmd{ifdefprotected}\see{aut:tst:def}
    \item 添加了 \cmd{ifcsprotected}\see{aut:tst:def}
    \item 添加了 \cmd{ifdefltxprotect}\see{aut:tst:def}
    \item 添加了 \cmd{ifcsltxprotect}\see{aut:tst:def}
    \item 添加了 \cmd{ifdefempty}\see{aut:tst:def}
    \item 添加了 \cmd{ifcsempty}\see{aut:tst:def}
    \item 改进了 \cmd{ifdefvoid}\see{aut:tst:def}
    \item 改进了 \cmd{ifcsvoid}\see{aut:tst:def}
    \item 添加了 \cmd{ifstrempty}\see{aut:tst:str}
    \item 添加了 \cmd{setbool}\see{aut:bo1:bol}
    \item 添加了 \cmd{settoggle}\see{aut:bo1:tgl}
  \end{release}

  \begin{release}{1.5}{2008-04-26}
    \item 添加了 \cmd{defcounter}\see{use:cal}
    \item 添加了 \cmd{deflength}\see{use:cal}
    \item 添加了 \cmd{ifdefstring}\see{aut:tst:def}
    \item 添加了 \cmd{ifcsstring}\see{aut:tst:def}
    \item 改进了 \cmd{rmntonum}\see{aut:msc}
    \item 添加了 \cmd{ifrmnum}\see{aut:msc}
    \item 添加了扩展的 \pdf 书签到本手册
    \item 稍微重新排列了手册内容
  \end{release}

  \begin{release}{1.4}{2008-01-24}
    \item 解决了与 \sty{tex4ht} 的冲突
  \end{release}

  \begin{release}{1.3}{2007-10-08}
    \item 将 \sty{elatex} 包重命名为 \sty{etoolbox}\see{int}
    \item 将 \cmd{newswitch} 重命名为 \cmd{newtoggle}(名称冲突)\see{aut:bo1:tgl}
    \item 将 \cmd{provideswitch} 重命名为 \cmd{providetoggle}(保持一致性)\see{aut:bo1:tgl}
    \item 将 \cmd{switchtrue} 重命名为 \cmd{toggletrue}(保持一致性)\see{aut:bo1:tgl}
    \item 将 \cmd{switchfalse} 重命名为 \cmd{togglefalse}(保持一致性)\see{aut:bo1:tgl}
    \item 将 \cmd{ifswitch} 重命名为 \cmd{iftoggle}(保持一致性)\see{aut:bo1:tgl}
    \item 将 \cmd{notswitch} 重命名为 \cmd{nottoggle}(保持一致性)\see{aut:bo1:tgl}
    \item 添加了 \cmd{AtEndPreamble}\see{use:pre}
    \item 添加了 \cmd{AfterEndDocument}\see{use:pre}
    \item 添加了 \cmd{AfterPreamble}\see{use:pre}
    \item 添加了 \cmd{undef}\see{aut:def:def}
    \item 添加了 \cmd{csundef}\see{aut:def:def}
    \item 添加了 \cmd{ifdefvoid}\see{aut:tst:def}
    \item 添加了 \cmd{ifcsvoid}\see{aut:tst:def}
    \item 添加了 \cmd{ifdefequal}\see{aut:tst:def}
    \item 添加了 \cmd{ifcsequal}\see{aut:tst:def}
    \item 添加了 \cmd{ifstrequal}\see{aut:tst:str}
    \item 添加了 \cmd{listadd}\see{aut:lst:int}
    \item 添加了 \cmd{listeadd}\see{aut:lst:int}
    \item 添加了 \cmd{listgadd}\see{aut:lst:int}
    \item 添加了 \cmd{listxadd}\see{aut:lst:int}
    \item 添加了 \cmd{listcsadd}\see{aut:lst:int}
    \item 添加了 \cmd{listcseadd}\see{aut:lst:int}
    \item 添加了 \cmd{listcsgadd}\see{aut:lst:int}
    \item 添加了 \cmd{listcsxadd}\see{aut:lst:int}
    \item 添加了 \cmd{ifinlist}\see{aut:lst:int}
    \item 添加了 \cmd{xifinlist}\see{aut:lst:int}
    \item 添加了 \cmd{ifinlistcs}\see{aut:lst:int}
    \item 添加了 \cmd{xifinlistcs}\see{aut:lst:int}
    \item 添加了 \cmd{dolistloop}\see{aut:lst:int}
    \item 添加了 \cmd{dolistcsloop}\see{aut:lst:int}

  \end{release}

  \begin{release}{1.2}{2007-07-13}

    \item 将 \cmd{patchcommand} 重命名为 \cmd{patchcmd}(名称冲突)\see{aut:pat}
    \item 将 \cmd{apptocommand} 重命名为 \cmd{apptocmd}(保持一致性)\see{aut:pat}
    \item 将 \cmd{pretocommand} 重命名为 \cmd{pretocmd}(保持一致性)\see{aut:pat}
    \item 添加了 \cmd{newbool}\see{aut:bo1:bol}
    \item 添加了 \cmd{providebool}\see{aut:bo1:bol}
    \item 添加了 \cmd{booltrue}\see{aut:bo1:bol}
    \item 添加了 \cmd{boolfalse}\see{aut:bo1:bol}
    \item 添加了 \cmd{ifbool}\see{aut:bo1:bol}
    \item 添加了 \cmd{notbool}\see{aut:bo1:bol}
    \item 添加了 \cmd{newswitch}\see{aut:bo1:tgl}
    \item 添加了 \cmd{provideswitch}\see{aut:bo1:tgl}
    \item 添加了 \cmd{switchtrue}\see{aut:bo1:tgl}
    \item 添加了 \cmd{switchfalse}\see{aut:bo1:tgl}
    \item 添加了 \cmd{ifswitch}\see{aut:bo1:tgl}
    \item 添加了 \cmd{notswitch}\see{aut:bo1:tgl}
    \item 添加了 \cmd{DeclareListParser}\see{aut:lst:inp}
    \item 添加了 \cmd{docsvlist}\see{aut:lst:inp}
    \item 添加了 \cmd{rmntonum}\see{aut:msc}

  \end{release}

  \begin{release}{1.1}{2007-05-28}

    \item 添加了 \cmd{protected@csedef}\see{aut:def:def}
    \item 添加了 \cmd{protected@csxdef}\see{aut:def:def}
    \item 添加了 \cmd{gluedef}\see{aut:def:cal}
    \item 添加了 \cmd{gluegdef}\see{aut:def:cal}
    \item 添加了 \cmd{csgluedef}\see{aut:def:cal}
    \item 添加了 \cmd{csgluegdef}\see{aut:def:cal}
    \item 添加了 \cmd{mudef}\see{aut:def:cal}
    \item 添加了 \cmd{mugdef}\see{aut:def:cal}
    \item 添加了 \cmd{csmudef}\see{aut:def:cal}
    \item 添加了 \cmd{csmugdef}\see{aut:def:cal}
    \item 添加了 \cmd{protected@eappto}\see{aut:hok:app}
    \item 添加了 \cmd{protected@xappto}\see{aut:hok:app}
    \item 添加了 \cmd{protected@cseappto}\see{aut:hok:app}
    \item 添加了 \cmd{protected@csxappto}\see{aut:hok:app}
    \item 添加了 \cmd{protected@epreto}\see{aut:hok:pre}
    \item 添加了 \cmd{protected@xpreto}\see{aut:hok:pre}
    \item 添加了 \cmd{protected@csepreto}\see{aut:hok:pre}
    \item 添加了 \cmd{protected@csxpreto}\see{aut:hok:pre}
    \item 修复了 \cmd{newrobustcmd} 中的 bug\see{use:def}
    \item 修复了 \cmd{renewrobustcmd} 中的 bug\see{use:def}
    \item 修复了 \cmd{providerobustcmd} 中的 bug\see{use:def}

  \end{release}

  \begin{release}{1.0}{2007-05-07}

    \item 初始公开版本

  \end{release}

\end{changelog}

\end{document}
