% \iffalse meta-comment
%
% Copyright (C) 1993-2022
% The LaTeX Project and any individual authors listed elsewhere
% in this file.
%
% This file is part of the LaTeX base system.
% -------------------------------------------
%
% It may be distributed and/or modified under the
% conditions of the LaTeX Project Public License, either version 1.3c
% of this license or (at your option) any later version.
% The latest version of this license is in
%    https://www.latex-project.org/lppl.txt
% and version 1.3c or later is part of all distributions of LaTeX
% version 2008 or later.
%
% This file has the LPPL maintenance status "maintained".
%
% The list of all files belonging to the LaTeX base distribution is
% given in the file `manifest.txt'. See also `legal.txt' for additional
% information.
%
% The list of derived (unpacked) files belonging to the distribution
% and covered by LPPL is defined by the unpacking scripts (with
% extension .ins) which are part of the distribution.
%
% \fi
% \iffalse
%
%<class>\NeedsTeXFormat{LaTeX2e}
%<class>\ProvidesClass{ltxdoc}
%<class>         [2022/06/22 v2.1h Standard LaTeX documentation class]
%
%<*driver>
\documentclass{ltxdoc}
\GetFileInfo{ltxdoc.cls}
\providecommand\dst{\expandafter{\normalfont\scshape docstrip}}

\usepackage[T1]{fontenc}
\usepackage{tabularray}
\usepackage{caption}

%%%%%%%%%%%%% 以下设置中文字体 %%%%%%%%%%%%%%%%%%%%%%%%%%%%%%%%%%%%%%%%%
\usepackage[fontset=source]{ctex}

%^^A \newCJKfontfamily[song]\songti{SourceHanSerifSC-regular.otf} %%自定义宋体 \sognti ,作为文章中文主字体
\newCJKfontfamily[Hei]\Heiti{SourceHanSansSC-regular.otf} %%自定义黑体 \heiti ,在幻灯片中黑体(SimHei)最漂亮
%^^A \newCJKfontfamily[hei]\heiti{SourceHanSansSC-Light.otf} %%自定义黑体 \heiti ,在幻灯片中黑体(SimHei)最漂亮
\newCJKfontfamily[kai]\kaiti{FandolKai-regular.otf} %%设置中文字体楷体 \kaiti ,用于强调
\newCJKfontfamily[heitixt]\heitixt{思源黑体_CN_LightItalic.ttf} %% 自定义\heitixt命令,显示思源黑体斜体
%%%%%%%%%%%%% 以上设置中文字体 %%%%%%%%%%%%%%%%%%%%%%%%%%%%%%%%%%%%%%%%%

%%%%%%%%%%%%% 以下设置中文版式 %%%%%%%%%%%%%%%%%%%%%%%%%%%%%%%%%%%%%%%%%
\usepackage{indentfirst} %%% 首行缩进
\setlength{\parindent}{2em} %%% 缩进2个字符(中文为2个字)
\linespread{1.35} %%% 设置行间距
%%%%%%%%%%%%% 以上设置中文版式 %%%%%%%%%%%%%%%%%%%%%%%%%%%%%%%%%%%%%%%%%


%%%%%%%%%%%% 以下设置书签、目录 %%%%%%%%%%%%%%%%%%%%%%%%%%%%%%%%%%%%%%%%
\usepackage{xcolor}
\usepackage[colorlinks=true,linkcolor=red]{hyperref}
%%%%%%%%%%%% 以上设置书签、目录 %%%%%%%%%%%%%%%%%%%%%%%%%%%%%%%%%%%%%%%%%


%%%%%%% 以下在 tabular 表格中定制 横线如\hlinew{1.2pt} %%%%%%
\makeatletter
\def\hlinew#1{%
\noalign{\ifnum0=`}\fi\hrule \@height #1 \futurelet
\reserved@a\@xhline}
\makeatother%
%%%%%%% 以上在 tabular 表格中定制 横线如\hlinew{1.2pt} %%%%%%

%%%%%%% 以下自定义脚注 %%%%%%%%%%%%%%%%%%%%%%%%%%%%%%%%%%%%
\setlength{\footnotesep}{0.5cm} %%%设置几第脚注之间的距离
\setlength{\skip\footins}{3em} %%%设置脚注与正文之间的距离
%%\renewcommand\footnoterule{} %%%定义脚注线为空
\renewcommand\footnoterule{
     \kern -3pt                         % This -3 is negative
     \hrule width 0.6\textwidth height 0.6pt % of the sum of this 1
     \kern 2pt} %%%
%%%%%%% 以上自定义脚注 %%%%%%%%%%%%%%%%%%%%%%%%%%%%%%%%%%%%

%%%%%%% 以下制作索引 %%%%%%%%%%%%%%%%%%%%%%%%%%%%%%%%%%%%
\usepackage {makeidx}
\makeindex
%%%%%%% 以上制作索引 %%%%%%%%%%%%%%%%%%%%%%%%%%%%%%%%%%%%


\renewcommand{\contentsname}{\heiti 目\ 录}   %%% 在{document}后面加入该命令,将"contents"变成“目  录”
\renewcommand{\refname}{\heiti 参考文献}
\renewcommand{\tablename}{表}
\renewcommand{\abstractname}{\heiti 摘\ 要}

\title{{\huge \Heiti 在}\ {\huge \textbf{\LaTeXe}}\ {\huge \Heiti 中使用} \\[2pt]
       {\Huge \textbf{\texttt{ltxdoc.dtx}}}\ {\huge \Heiti 文件}\
        \thanks{该文件的版本号为 \fileversion,最新修订日期为 \filedate。}\\[4pt]
        包含\ \texttt{ltxdoc.cls}\ 的代码}
\date{\filedate}
\author{David Carlisle\\[4pt] 黄旭华\ \ \ \ 翻译}

\begin{document}
 \MaintainedByLaTeXTeam{latex}
 \maketitle
  \tableofcontents
\newpage
 \DocInput{ltxdoc-zh-cn.dtx}
\end{document}
%</driver>
%
% \fi
%
% \providecommand\pkg[1]{\texttt{#1}}
%
%
% \changes{v2.0i}{1994/04/29}{Update the documentation.}
% \changes{v2.0s}{1998/08/17}{(RmS) Documentation fixes.}
% \changes{v2.1a}{2020/08/21}{Integration of new hook management interface}
%
% \section[\LaTeX\ {\heiti 的源文档}]{\LaTeX\ {\heiti 的源文档}}
%
% 该类文件(class file)用于记录 \LaTeX\ 的源文件(source files)。然而,您可能会发现
% 它通常作为一个类(class)用于排版以“doc”格式生成的文件的文档(documentation)。
%
% 标准发行版(standard distribution)中的每个文档文件(documented file)都带有扩展名|dtx|。
% docstrip 系统将从源代码中提取适当的类包(class package)或 initex 文件。可以
% 使用 \LaTeXe\ 直接处理每个 |dtx| 文件,例如:
%\begin{verbatim}
% latex2e docclass.dtx
%\end{verbatim}
% 将生成 Class 和 package 接口(interface)的文档。
%
% 如果您对 |sources2e.tex| 文件进行了 \LaTeX\ 编译,则用于生成 \LaTeXe\ 格式的每个文件
% (即不包括标准类和包)将一起打印在一个文档中。这样做的好处是,可以生成所有源文件的
% 宏使用情况(macro usage)的完整索引。
%
% 如果需要自定义这些文件的排版,有两个选择:
% \begin{itemize}
% \item 在输入源文件(source file)之前,您可以使用 \dst\ 和“driver”模块(module)来提取
% 一个小的 \LaTeX\ 文件,可以编辑该文件以使用所需的任何类或包的选项。
% \item 您可以创建一个 |ltxdoc.cfg| 文件。无论何时使用 |ltxdoc| 类,都会读取此配置文件,
% 因此可以用该配置文件来自定义所有源文件(source files)的排版,而无需编辑许多
% 小驱动程序文件(small driver files)。
%\end{itemize}
%
% 第二种选择通常更方便。下一节将讨论各种可能性。
%
% \section{\heiti 定制}
%
% 最简单的定制(customisation)形式是将更多选项传递给 |ltxdoc| 加载的 |article| 类。
% 例如,如果希望将所有文档格式化为 A4 纸,请在 |ltxdoc.cfg| 中添加下面这一行:
%\begin{verbatim}
% \PassOptionsToClass{a4paper}{article}
%\end{verbatim}
%
% 所有源文件(source files)分为两部分,以 |\MaybeStop| 分隔。第一部分(应该)包含“user(用户)”文档。
% 第二部分是源代码(source code)的完整文档列表。|doc| 宏包提供了 |\OnlyDescription| 命令,
% 该命令禁止显示代码列表。这也可以在配置文件中使用,但由于稍后读取 |doc| 包,因此必须
% 必须将 |\OnlyDescription| 的执行(execution)延迟到读取 |doc| 包之后。最简单的方法是使用 |\AtBeginDocument|。
% 因此,您可以将以下内容放在 |ltxdoc.cfg| 中。
%\begin{verbatim}
% \AtBeginDocument{\OnlyDescription}
%\end{verbatim}
%
% 如果您的文档依赖于使用旧版的 \pkg{doc},则可以通过传递 \texttt{doc2}\ 选项来
% 请求类加载 \pkg{doc}\ 2版。
%
% 如果处理了完整的源代码清单 |sources2e.tex|,那么默认情况下会生成索引(index)和
% 更新历史记录(change history),但是通常不会为单个文件生成索引。
%
% 例如,考虑 |ltclass.dtx|,它包含新类和包接口命令的源代码。在没有 |cfg| 文件的情况下,
% 将生成一个~19~页的文档。使用上面的配置,就生成一个可读性稍微好一点的文档(4~页)。
%
% 相反,如果您真的想详细阅读源代码清单,您会想要一个索引。同样,可以使用 doc 包提供的
% 索引命令,但必须延迟其执行。
%\begin{verbatim}
% \AtBeginDocument{\CodelineIndex\EnableCrossrefs}
% \AtEndDocument{\PrintIndex}
%\end{verbatim}
%
% |doc| 包使用 MakeIndex 和 |gind| 样式写入要排序的索引文件(index files),因此可以使用以下命令
%\begin{verbatim}
% makeindex -s gind.ist ltclass.idx
%\end{verbatim}
% 然后重新运行 \LaTeX
%
% 与打印更改历史记录(print a Change history)类似,您可以添加
%\begin{verbatim}
% \AtBeginDocument{\RecordChanges}
% \AtEndDocument{\PrintChanges}
%\end{verbatim}
% 到 |ltxdoc.cfg|,并将 MakeIndex 与以下命令一起使用
%\begin{verbatim}
% makeindex -s gglo.ist -o ltclass.gls ltclass.glo
%\end{verbatim}
%
% 最后,如果您不想列出 |source2e.tex| 的所有节(sections),则只能在 |cfg| 文件中使用 |\includeonly|:
%\begin{verbatim}
% \includeonly{ltvers,ltboxes}
%\end{verbatim}
%
% \MaybeStop{}
%
%
% \section{\heiti 选项}
%
%    \begin{macrocode}
%<*class>
%    \end{macrocode}
%
%    \begin{macrocode}
\DeclareOption{a5paper}{\@latexerr{Option not supported}%
   {}}
%    \end{macrocode}
%    阻止加载配置文件(config file)。
%    \begin{macrocode}
\newif\ifltxdoc@load@cfg@ \ltxdoc@load@cfg@true
\DeclareOption{nocfg}{\ltxdoc@load@cfg@false}
%    \end{macrocode}
%
%    支持将 \pkg{doc}\ 回滚到版本 2:
% \changes{v2.1e}{2022/05/27}{Support rolling  back doc to version 2}
%    \begin{macrocode}
\let\ltxdoc@doc@version\@empty  % 默认情况下使用当前版本
\DeclareOption{doc2}{\def\ltxdoc@doc@version{=v2}}
%    \end{macrocode}
%
%    \begin{macrocode}
\DeclareOption*{%
    \PassOptionsToClass  {\CurrentOption}{article}}
%    \end{macrocode}
%
% \section{\heiti 选项处理}
%
%    \begin{macrocode}
\ProcessOptions
%    \end{macrocode}
%
% \section{\heiti 局部配置}
% 导入局部配置文件(如果存在)。
%    \begin{macrocode}
\ifltxdoc@load@cfg@
\InputIfFileExists{ltxdoc.cfg}
           {\typeout{*************************************^^J%
                     * 使用局部配置文件 ltxdoc.cfg        ^^J%
                     *************************************}}
           {}
\else
   \typeout{*************************************^^J%
            * 忽略局部配置文件                   ^^J%
            *************************************}
\fi
%    \end{macrocode}
%
%
% \section{\heiti 加载 article 和 doc}
%
%    \begin{macrocode}
\LoadClass{article}
%    \end{macrocode}
%
%    默认情况下,加载当前 \pkg{doc}\ 版本(\cs{ltxdoc@doc@version}\ 为空)。如果
%    给定选项 \texttt{doc2},则加载版本2(\cs{ltxdoc@doc@version}\ 包含 \texttt{=v2})。
% \changes{v2.1e}{2022/05/27}{Support rolling  back doc to version 2}
%    \begin{macrocode}
\RequirePackage{doc}[\ltxdoc@doc@version]
%    \end{macrocode}
%
% 使 \verb+|+ 成为“短 verb”字符,但不能在文档前言中,因为活动字符(active character)可能会干扰加载的包。
%    \begin{macrocode}
\AtBeginDocument{\MakeShortVerb{\|}}
%    \end{macrocode}
%
% 由于“doc”文档往往有很多等宽材料(monospaced material),请设置一些 |tt| 替换
% (substitutions)以便静默地进行。
% \changes{v2.0p}{1995/11/02}{Add font substitutions}
% \changes{v2.0t}{1999/04/17}{Replaced octal number, CAR}
%    \begin{macrocode}
\DeclareFontShape{OT1}{cmtt}{bx}{n}{<-> ssub * cmtt/m/n}{}
\DeclareFontFamily{OMS}{cmtt}{\skewchar\font 48}  % '60
\DeclareFontShape{OMS}{cmtt}{m}{n}{<-> ssub * cmsy/m/n}{}
\DeclareFontShape{OMS}{cmtt}{bx}{n}{<-> ssub * cmsy/b/n}{}
%    \end{macrocode}
% 此替换(substitution)位于标准 fd 文件中,但不是静默的(silent)。
%    \begin{macrocode}
\DeclareFontShape{OT1}{cmss}{m}{it}{<->ssub*cmss/m/sl}{}
%    \end{macrocode}
%
%    \begin{macrocode}
\CodelineNumbered
\DisableCrossrefs
%    \end{macrocode}
%
% 稍微增加文本宽度,以便在 |macrocode| 环境中使用标准字体(standard fonts)以显示 72 栏代码。
% \changes{v2.0c}{1994/03/15}{Set \cs{textwidth}.}
%    \begin{macrocode}
\setlength{\textwidth}{355pt}
%    \end{macrocode}
%
% 对于较长的命令名,请稍微增加边距宽度(marginpar width)。并将左侧边距(left margin)增加相似的量。
% \changes{v2.0l}
%      {1994/05/25}{Increase \cs{marginparwidth}}
% \changes{v2.0q}{1995/11/28}
%      {Increase \cs{marginparwidth} and page margin.}
%    \begin{macrocode}
\addtolength\marginparwidth{30pt}
\addtolength\oddsidemargin{20pt}
\addtolength\evensidemargin{20pt}
%    \end{macrocode}
%
%
%    \begin{macrocode}
\setcounter{StandardModuleDepth}{1}
%    \end{macrocode}
%
% \section{\heiti 有用的缩写}
%
%   |\cmd{\foo}| 按原样打印 |\foo|。它可以在移动参数(moving arguments)中使用。它{\kaiti 不能}用于记录定义为“|\outer|”的命令,
%   也不能用于诸如 |\iftrue| 或由 |\newif| 定义的条件。\texttt{doc}\ 包中已经提供的 |\cs{foo}| 也会为喜欢这种语法的人打印 |\foo|。
%   (第二种形式可用于记录所有类型的命令,因此上述限制不适用)
% \begin{macro}{\cmd}
% \changes{v2.0k}{1994/05/21}{New definition, so \cmd\{ works.}
% \begin{macro}{\cs}
% \changes{v2.0d}{1994/03/17}{Add \cs{cs}}
% \changes{v2.1h}{2022/06/22}{Update \cs{cmd} definition to work with new \cs{cs}}
%    \begin{macrocode}
%\DeclareRobustCommand\cs[1]...                  %稍后定义
%\def\cmd#1{\cs{\expandafter\cmd@to@cs\string#1}}%不能与新的\cs一起用
\def\cmd#1{\texttt{\char`\\\expandafter\cmd@to@cs\string#1}}
\def\cmd@to@cs#1#2{\char\number`#2\relax}
%    \end{macrocode}
% \end{macro}
% \end{macro}
%
% \changes{v2.0r}{1996/01/11}{Removed \cs{star} since useless pr/2039}
%
% \begin{macro}{\marg}
% \changes{v2.0d}{1994/03/17}{Add \cs{marg}}
%    |\marg{text}| 打印 \marg{text},“强制参数(mandatory argument)” 。
%    \begin{macrocode}
\providecommand\marg[1]{%
  {\ttfamily\char`\{}\meta{#1}{\ttfamily\char`\}}}
%    \end{macrocode}
% \end{macro}
%
% \begin{macro}{\oarg}
%    |\oarg{text}| 打印 \oarg{text},“可选参数(optional argument)”。
%    \begin{macrocode}
\providecommand\oarg[1]{%
  {\ttfamily[}\meta{#1}{\ttfamily]}}
%    \end{macrocode}
% \end{macro}
%
% \begin{macro}{\parg}
%    |\parg{te,xt}| 打印 \parg{te,xt},“图像模式参数(picture mode argument)”。
% \changes{v2.0h}{1994/04/28}{Add \cs{parg}}
% \changes{v2.0o}{1995/08/09}{Use \cs{meta} when showing arguments}
%    \begin{macrocode}
\providecommand\parg[1]{%
  {\ttfamily(}\meta{#1}{\ttfamily)}}
%    \end{macrocode}
% \end{macro}
%
%
% \section{\heiti 旧注释}
%
% \LaTeXe\ 源包含大量继承自 \LaTeX2.09\ 的代码。此代码中的注释(comments)不是为了排版而设计的,
% 并且不包含必要的 \LaTeX\ 标记。\texttt{oldcomments}\ 环境对这些注释进行排版,自动检测
% 何时出现控制序列(control sequence),并隐式(implicitly)添加 |\verb|。这个过程不会产生
% 特别漂亮的页面,但它允许我们完整地记录新节(new sections),并对所有旧代码进行
% 某种形式的排版注释(typeset comments)。
% \changes{v2.0e}{1994/03/18}{Use a fixed font.}
%
% 扫描控件名称(control names)并将其放入 tt。实际上会(错误地)扫描以前的 |\\|,但这并不重要,
% 因为在实践中,这几乎从来没有后面跟着字母(letter)。(即 |\\foo|)会将 |foo| 放在 |\ttfamily| 中。
%    \begin{macrocode}
\def\oc@scan#1{%
  \ifx\oc@bslash#1%
                      \egroup\let\next\oc@bslash\else
  \ifcat a\noexpand#1%
                      #1\let\next\oc@scan\else
  \ifx\oc@percent#1%
                      \def\next{\char`\%\egroup}%
  \else
                      #1\let\next\egroup
  \fi\fi\fi\next}
%    \end{macrocode}
%
%    \begin{macrocode}
\def\oc@bslash{\bgroup\oc@ttf\char`\\\oc@scan}%
%    \end{macrocode}
%
%    \begin{macrocode}
\def\oc@verb#1{%
  \catcode`#1\active
  \uccode`\~`#1%
  \uppercase{\def~{{\oc@ttf\char`#1}}}}
%    \end{macrocode}
%
%    \begin{macrocode}
\begingroup
  \obeyspaces%
  \catcode`\/=\catcode`\\
  /catcode`/\/active
  /catcode`<=/catcode`{%
  /catcode`>=/catcode`}%
  /catcode`/{/active%
  /catcode`/}/active%
  /gdef/oldc< \end{oldcomments}>%
  /gdef/begmac<    \begin{macrocode}>%
  /gdef/obs</def <</oc@ttf/ >>>%
/endgroup%
%    \end{macrocode}
%
% \changes{v2.0y}{2019/09/16}{Say that oldcomments may not be accurate (gh/180)}
%    \begin{macrocode}
\begingroup
  \catcode`\/=\catcode`\\
  \catcode`\\=13
  /catcode`/|=/catcode`/%
  /catcode`/%=13
  /gdef/oldcomments{|
    /makeatletter
    /let/do/oc@verb/dospecials
    /frenchspacing/@vobeyspaces/obs
    /raggedright
    /oc@verb/>|
    /oc@verb/<|
    /let\/oc@bslash
    /let%/oc@percent
    /obeylines
    /parindent/z@
    /ttfamily/expandafter/let/expandafter/oc@ttf/the/font
    /rmfamily
    /textit{Historical /LaTeX/,2.09 comments (not necessarily accurate any more):}
    /hfuzz/maxdimen
    }
/endgroup
%    \end{macrocode}
%
%    \begin{macrocode}
\begingroup
  \sloppy%
  \obeylines%
  \gdef\oc@percent#1^^M{%
    \ifvmode%
    \def\commentline{#1}%
    \ifx\commentline\oldc%
    \textit{End of historical \LaTeX\,2.09 comments.}
    \end{oldcomments}%
    \else%
    \ifx\commentline\begmac%
    \begin{macrocode}%
    \else%
    \leavevmode%
    #1^^M%
    \fi\fi%
    \else%
    {\oc@ttf\char`\%}#1^^M%
    \fi}%
\endgroup%
%    \end{macrocode}
%
%
% \section{DocInclude}
%
%    \begin{macrocode}
\@addtoreset{CodelineNo}{part}
%    \end{macrocode}
%
% \begin{macro}{\DocInclude}
% 或多或少与 |\include| 完全相同,但在 |dtx| 文件上使用 |\DocInput|,而在 |tex| 文件上不使用 |\input|。
% \changes{v2.0b}{1994/03/14}{Rename from \cs{docinclude}}
% \changes{v2.0f}{1994/03/25}{Use \cs{part}}
% \changes{v2.0u}{1999/08/08}{Also works for .fdd (M. Schroeder)}
%    \begin{macrocode}
\def\partname{File}
%    \end{macrocode}
%
%
%
% \changes{v2.0z}{2020/05/17}{Support spaces \emph{within} filenames
%    (gh/218)}
%    \begin{macrocode}
\newcommand*{\DocInclude}[1]{%
  \relax
  \clearpage
  \docincludeaux
  \IfFileExists{#1.fdd}%
    {\def\currentfile{#1.fdd}}%
    {\def\currentfile{#1.dtx}}%
  \ifnum\@auxout=\@partaux
    \@latexerr{\string\include\space cannot be nested}\@eha
  \else
    \set@curr@file{#1}%
    \edef\@curr@file{\@strip@tex@ext\@curr@file}%
    \expandafter\@docinclude\expandafter{\@curr@file}
 \fi}
\def\@docinclude#1 {\clearpage
%    \end{macrocode}
%
% \changes{v2.1bj}{2020/10/04}{Quotes around the aux file name removed,
%                       they are no longer needed and upset BibTeX (gh/400)}
%    \begin{macrocode}
\if@filesw \immediate\write\@mainaux{\string\@input{#1.aux}}\fi
\@tempswatrue\if@partsw \@tempswafalse\edef\@tempb{#1}\@for
\@tempa:=\@partlist\do{\ifx\@tempa\@tempb\@tempswatrue\fi}\fi
\if@tempswa \let\@auxout\@partaux \if@filesw
\immediate\openout\@partaux "#1.aux"
\immediate\write\@partaux{\relax}\fi
\@filehook@set@CurrentFile
%    \end{macrocode}
% 我们需要保存(并稍后恢复)与索引相关的各种命令,这些命令可能会被包含的文件更改。
%    \begin{macrocode}
\let\@ltxdoc@PrintIndex\PrintIndex
\let\PrintIndex\relax
\let\@ltxdoc@PrintChanges\PrintChanges
\let\PrintChanges\relax
\let\@ltxdoc@theglossary\theglossary
\let\@ltxdoc@endtheglossary\endtheglossary
\part{\currentfile}%
  {\let\ttfamily\relax
  \xdef\filekey{\filekey, \thepart={\ttfamily\currentfile}}}%
\DocInput{\currentfile}%
\let\PrintIndex\@ltxdoc@PrintIndex
\let\PrintChanges\@ltxdoc@PrintChanges
\let\theglossary\@ltxdoc@theglossary
\let\endtheglossary\@ltxdoc@endtheglossary
\clearpage
\@writeckpt{#1}\if@filesw \immediate\closeout\@partaux \fi
\else\@nameuse{cp@#1}\fi\let\@auxout\@mainaux}
%    \end{macrocode}
%
% \changes{v2.0x}{2018/03/15}{Set up \cs{protect} in index entries}
%    \begin{macrocode}
\gdef\codeline@wrindex#1{\if@filesw
%    \end{macrocode}
%   在索引项(index entries)中将 \cs{protect} 设置为合适的值(我们不能使用 |\set@display@protect|,
%   因为这会导致命令后出现不同的空格数,具体取决于写入索引之前发生的展开的数量)。
%    \begin{macrocode}
      \begingroup
        \let\protect\noexpand
        \immediate\write\@indexfile
            {\string\indexentry{#1}%
            {\filesep\number\c@CodelineNo}}%
      \endgroup\fi}
%    \end{macrocode}
% \end{macro}
%
%    \begin{macrocode}
\let\filesep\@empty
%    \end{macrocode}
%
%
%  \begin{macro}{\aalph}
% 特殊形式的 |\alph| 作为当前 |source2e.tex| 包含 26 个以上的文件。
% \changes{v2.0n}{1994/05/27}{Use uppercase letters, for makeindex}.
%    \begin{macrocode}
\def\aalph#1{\@aalph{\csname c@#1\endcsname}}
\def\@aalph#1{%
  \ifcase#1\or a\or b\or c\or d\or e\or f\or g\or h\or i\or
         j\or k\or l\or m\or n\or o\or p\or q\or r\or s\or
         t\or u\or v\or w\or x\or y\or z\or A\or B\or C\or
         D\or E\or F\or G\or H\or I\or J\or K\or L\or M\or
         N\or O\or P\or Q\or R\or S\or T\or U\or V\or W\or
         X\or Y\or Z\else\@ctrerr\fi}
%    \end{macrocode}
%  \end{macro}
%
% \begin{macro}{\docincludeaux}
% \changes{v2.06}{1994/03/31}{Use \cs{footnotesize} in file key.}
% \changes{v2.0k}{1994/05/21}{Use \cs{aalph}}
% \changes{v2.0v}{2017/03/13}{Use \cs{parbox}[t] in file
%         key to maintain space between the text block and the page foot.}
%    \begin{macrocode}
\def\docincludeaux{%
  \def\thepart{\aalph{part}}\def\filesep{\thepart-}%
  \let\filekey\@gobble
  \g@addto@macro\index@prologue{%
    \gdef\@oddfoot{\parbox[t]{\textwidth}{\strut\footnotesize
       \raggedright{\bfseries File Key:} \filekey}}%
    \let\@evenfoot\@oddfoot}%
  \global\let\docincludeaux\relax
 \gdef\@oddfoot{%
   \expandafter\ifx\csname ver@\currentfile\endcsname\relax
    File \thepart: {\ttfamily\currentfile} %
   \else
    \GetFileInfo{\currentfile}%
    File \thepart: {\ttfamily\filename} %
    Date: \filedate\ %
    Version \fileversion
    \fi
    \hfill\thepage}%
 \let\@evenfoot\@oddfoot}%
%    \end{macrocode}
% \end{macro}
%
% \begin{macro}{\MaintainedByLaTeXTeam}
% \changes{v2.0v}{2015/03/25}{macro added}
% \changes{v2.0w}{2015/03/25}{use display block not footnote text}
% 生成对错误数据库(bug database)的样板引用(boilerplate reference)。
%    \begin{macrocode}
\def\MaintainedBy#1{\gdef\@maintainedby{#1}}
%    \end{macrocode}
%
%    \begin{macrocode}
\let\@maintainedby\@empty
%    \end{macrocode}
%
%    \begin{macrocode}
\def\MaintainedByLaTeXTeam#1{%
{\gdef\@maintainedby{%
此文件由 \LaTeX{} 项目团队维护。\\%
可以在以下位置打开 Bug 报告(类别 \texttt{#1})\\%
\url{https://latex-project.org/bugs.html}.}}}
%    \end{macrocode}
%
%    \begin{macrocode}
\def\@maketitle{%
  \newpage
  \null
  \vskip 2em%
  \begin{center}%
  \let \footnote \thanks
    {\LARGE \@title \par}%
    \vskip 1.5em%
    {\large
      \lineskip .5em%
      \begin{tabular}[t]{c}%
        \@author
      \end{tabular}\par}%
    \vskip 1em%
    {\large \@date}%
    \ifx\@maintainedby\@empty
    \else
    \vskip 1em%
    \fbox{\fbox{\begin{tabular}{@{}l@{}}\@maintainedby\end{tabular}}}%
    \fi
  \end{center}%
  \par
  \vskip 1.5em}
%    \end{macrocode}
% \end{macro}
%
%
%    \begin{macrocode}
\def\task#1#2{}
%    \end{macrocode}
%
%
% \changes{v2.1f}{2022/06/03}{define \cs{pkg} and other commands}
% \changes{v2.1f}{2022/06/03}{allow \_ in macro names}
% l3doc.cls 的一些功能
% \changes{v2.1g}{2022/06/14}{define \cs{pkg} but delay until begin document}
%    \begin{macrocode}
\DeclareRobustCommand\cs[1]{\texttt{\bslash\detokenize{#1}}}%
\AtBeginDocument{%
  \renewcommand\PrintMacroName[1]{\MacroFont\detokenize{#1}}%
%    \end{macrocode}
%    如果有人加载了 \texttt{csquotes}\ 或在前言中做了一些定义,我们会提供这些延迟(delated)。
%    \begin{macrocode}
  \providecommand\LuaTeX{Lua\TeX}
  \providecommand\cls{\textsf}
  \providecommand\pkg{\textsf}
  \providecommand\enquote[1]{``#1''}
  \providecommand\url{\texttt}
}
%    \end{macrocode}
%
%    \begin{macrocode}
%</class>
%    \end{macrocode}
%
%
%
% \section{\heiti 配置文件}
%
%
%    \begin{macrocode}
%<*cfg>
%
%
% 这是我们用来格式化 LaTeX 内核源代码的 ltxdoc 配置文件。
%
%
% 版权所有2006、2007、2011 Heiko Oberdiek (海科·奥伯代克)
% 版权所有2014-2021 LaTeX\ 项目
%
\ProvidesFile{ltxdoc.cfg}%
  [2022/06/14 v2.0d ltxdoc.cls 配置 (LaTeX 项目)]
\PassOptionsToClass{a4paper}{article}
% hyperref 和 hypdoc 现在加载较晚(或由用户加载),因此我们必须等待一些调整,
% 直到完成为止
\AddToHook{package/hyperref/after}{%
%%  \RequirePackage{hypdoc}%       % 这现在由 doc 触发
  \RequirePackage{pdftexcmds}\relax
  \ifnum\pdf@strcmp{\jobname}{inputenc}=0 %
    \hypersetup{pdfencoding=auto}%
    \pdfstringdefDisableCommands{%
      \def\meta#1{% inputenc.dtx
        \9060\010#1\9060\011%
      }%
    }%
  \else
  \fi
  \pdfstringdefDisableCommands{%
    \let\env\relax % longtable.dtx
    \let\mytt\relax % tabularx.dtx
  }%
}
% 对于 LaTeX 内核之外的包文档来说,这应该可以很好地工作,但是如果不行,您可
% 以使用“nocfg”选项来阻止加载,即,
%
%   \documentclass[nocfg]{ltxdoc]
%
% 或者提供自己的配置文件
\endinput
%</cfg>
%    \end{macrocode}
% \Finale
%
