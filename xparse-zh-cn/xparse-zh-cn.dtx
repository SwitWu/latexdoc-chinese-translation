% \iffalse meta-comment
%
%% File: xparse.dtx
%
% Copyright (C) 1999 Frank Mittelbach, Chris Rowley, David Carlisle
%           (C) 2004-2008 Frank Mittelbach, The LaTeX Project
%           (C) 2009-2023 The LaTeX Project
%
% It may be distributed and/or modified under the conditions of the
% LaTeX Project Public License (LPPL), either version 1.3c of this
% license or (at your option) any later version.  The latest version
% of this license is in the file
%
%    https://www.latex-project.org/lppl.txt
%
% This file is part of the "l3packages bundle" (The Work in LPPL)
% and all files in that bundle must be distributed together.
%
% -----------------------------------------------------------------------
%
% The development version of the bundle can be found at
%
%    https://github.com/latex3/latex3
%
% for those people who are interested.
%
%<*driver|package>
% The version of expl3 required is tested as early as possible, as
% some really old versions do not define \ProvidesExplPackage.
\RequirePackage{expl3}[2018-04-12]
%<package>\@ifpackagelater{expl3}{2018-04-12}
%<package>  {}
%<package>  {%
%<package>    \PackageError{xparse}{Support package l3kernel too old}
%<package>      {%
%<package>        Please install an up to date version of l3kernel\MessageBreak
%<package>        using your TeX package manager or from CTAN.\MessageBreak
%<package>        \MessageBreak
%<package>        Loading xparse will abort!%
%<package>      }%
%<package>    \endinput
%<package>  }
%</driver|package>
%<*driver>
\documentclass[full]{l3doc}
% ^^A ======================== 中文版特定设置 ========================
\usepackage[fontset=source]{ctex}
\ctexset{
  contentsname = {\centerline{目\quad 录}}
}
\usepackage{xcolor}
\definecolor{spot}{HTML}{003399}
\definecolor{code}{HTML}{a25e26}
\definecolor{verb}{HTML}{007f00}
\let\OriginMF\MacroFont
\def\MacroFont{\color{spot}\OriginMF}
\pretocmd{\macrocode}{\color{code}\def\MacroFont{\color{code}\OriginMF}}{}{}
\AtBeginEnvironment{macrocode}{\def\MacroFont{\color{code}\OriginMF}}
\AtEndEnvironment{macrocode}{\def\MacroFont{\color{spot}\OriginMF}}
\AtBeginEnvironment{quote}{\color{verb}}
\AtBeginEnvironment{flushleft}{\color{verb}}
\AtBeginEnvironment{verbatim}{\color{verb}\let\MF\MacroFont\def\MacroFont{\color{verb}\OriginMF}}
\AtEndEnvironment{verbatim}{\let\MacroFont\MF}
\ExplSyntaxOn
\pretocmd{\__codedoc_typeset_functions:}{\color{spot}\arrayrulecolor{spot}}{}{}
\ExplSyntaxOff
\hypersetup{%
  colorlinks=true,
  linkcolor=spot,
  urlcolor=spot,
  citecolor=spot,
  bookmarksopen=false,
  bookmarksnumbered=false,
  plainpages=false,
  pdfview=FitH}
\IndexPrologue
  {
    \section*{索引}
    \markboth{索引}{索引}
    \addcontentsline{toc}{section}{索引}
    斜体数字指向相应条目描述的页面,
    下划线数字指向定义的代码行,
    其它的都指向使用条目的页面。
  }
% ^^A ======================== 中文版设置结束 ========================
%
  \usepackage{amstext}
\begin{document}
  \DocInput{\jobname.dtx}
\end{document}
%</driver>
% \fi
%
% \providecommand\acro[1]{\textsc{\MakeLowercase{#1}}}
% \newenvironment{arg-description}{%
%   \begin{itemize}\def\makelabel##1{\hss\llap{\bfseries##1}}}{\end{itemize}}
%
% \title{^^A
%   \color{spot}\bfseries{\huge\textsf{xparse} 宏包}\\\Large 文档命令解析器^^A
% }
%
% \author{^^A
%   \begin{tabular}{r@{\qquad}l}
%  The \LaTeX{} Project\thanks
%    {^^A
%      E-mail:
%        \href{mailto:latex-team@latex-project.org}
%          {latex-team@latex-project.org}^^A
%    }^^A
%   & 2023年10月10日
%   \\[2ex]
%   张泓知 & \zhtoday 翻译
%   \end{tabular}
% }
%
% \date{}
%
% \maketitle
%
% \addtocontents{toc}{\protect\begin{multicols}{2}}
%
% \tableofcontents
%
% \newpage
%
% \begin{documentation}
%
% \pkg{xparse}包提供了一个用于生成文档级命令的高级接口。因此,它旨在替代
% \LaTeXe{} 的 \cs{newcommand} 宏。然而,\pkg{xparse} 的工作方式是,函数的
% 接口(例如可选参数、星号和必选参数)与内部实现分离。\pkg{xparse}为函数的
% 内部形式提供了一个规范化的输入,独立于文档级别的参数排列。
%
% 目前,\pkg{xparse}中被认为是
% \enquote{稳定的 (stable) } 有:
% \begin{itemize}
%   \item \cs{NewDocumentCommand}\\
%     \cs{RenewDocumentCommand}\\
%     \cs{ProvideDocumentCommand}\\
%     \cs{DeclareDocumentCommand}
%   \item \cs{NewDocumentEnvironment}\\
%     \cs{RenewDocumentEnvironment}\\
%     \cs{ProvideDocumentEnvironment}\\
%     \cs{DeclareDocumentEnvironment}
%   \item \cs{NewExpandableDocumentCommand}\\
%     \cs{RenewExpandableDocumentCommand}\\
%     \cs{ProvideExpandableDocumentCommand}\\
%     \cs{DeclareExpandableDocumentCommand}
%   \item \cs{IfNoValue(TF)}
%   \item \cs{IfValue(TF)}
%   \item \cs{IfBoolean(TF)}
% \end{itemize}
% 除了目前被视为“实验性”的其他功能。请尝试这里提供的所有命令,但请注意,
% 实验性的功能可能会更改或消失。
%
% \section{指定参数}
%
% 在介绍用于创建文档命令的功能之前,将展示使用 \pkg{xparse}
% 指定参数的方法。为了允许每个参数独立定义,
% \pkg{xparse} 不仅需要知道函数的参数数量,还需要知道每个参数的性质。
% 这是通过构建一个\emph{参数规范}来完成的,该规范定义了参数的数量、每个参数的类型
% 和 \pkg{xparse} 读取用户输入并正确传递到内部函数所需的任何附加信息。
%
% 参数规范的基本形式是一个字母列表,其中每个字母定义了一个参数类型。
% 正如下面将要描述的那样,某些类型需要额外的信息,比如默认值。
% 参数类型可以分为两种:定义必须参数的类型(如果未找到可能会引发错误),
% 以及定义可选参数的类型。必须类型包括:
% \begin{itemize}[font=\ttfamily]
%   \item[m] 标准的必选参数,可以是单个标记,也可以是由花括号 |{}| 包围的多个标
%     记。无论输入是什么,参数都将被传递到内部代码,但不带有外部花括号。这是 \pkg{
%     xparse} 中用于普通 \TeX{} 参数的类型说明符。
%   \item[r] 给出为 \texttt{r}\meta{token1}\meta{token2},表示一个“必需”的分隔
%     参数,其分隔符为 \meta{token1} 和 \meta{token2}。如果缺少开始分隔符 
%     \meta{token1},将在适当的错误后插入默认标记 |-NoValue-|。
%   \item[R] 给出为 \texttt{R}\meta{token1}\meta{token2}\marg{default},这是一个
%     “必需”的分隔参数,与 \texttt{r} 类似,但具有用户可定义的恢复值 \meta{default}
%     ,而不是 |-NoValue-|。
%   \item[v] 以“verbatim”方式读取参数,介于以下字符和其下一个出现之间,类似于 \L
%     aTeXe{} 命令 \cs{verb} 的参数。因此,\texttt{v} 类型的参数在两个相同的字
%     符之间读取,但这些字符不能是 |%|、|\|、|#|、|{|、|}| 或 \verb*| |。verbatim
%     参数也可以被括在花括号 |{| 和 |}| 中。拥有 verbatim 参数的命令在另一个函
%     数的参数中出现时会产生错误。
%   \item[b] 只适用于环境的参数规范,表示环境的主体部分,位于 |\begin|
%     \marg{environment} 和 |\end|\marg{environment} 之间。详见第~\ref{sec:body} 节。
% \end{itemize}
% 定义可选参数的类型有:
% \begin{itemize}[font=\ttfamily]
%   \item[o] 一个标准的 \LaTeX{} 可选参数,用方括号括起来,如果未给出参数,
%     将提供特殊标记 |-NoValue-|(如后文所述)。
%   \item[d] 给定为 \texttt{d}\meta{token1}\meta{token2},一个由 \meta{token1}
%     和 \meta{token2} 分隔的可选参数。与 \texttt{o} 类似,如果未给出值,
%     将返回特殊标记 |-NoValue-|。
%   \item[O] 给定为 \texttt{O}\marg{default},与 \texttt{o} 类似,
%     但如果未给出值,则返回 \meta{default}。
%   \item[D] 给定为 \texttt{D}\meta{token1}\meta{token2}\marg{default},
%     与 \texttt{d} 类似,但如果未给出值,则返回 \meta{default}。
%     在内部,\texttt{o}、\texttt{d} 和 \texttt{O} 类型是构造 \texttt{D} 类型参数的快捷方式。
%   \item[s] 一个可选的星号,如果存在星号则结果为 \cs{BooleanTrue},否则为 \cs{BooleanFalse}
%     (如后文所述)。
%   \item[t] 一个可选的 \meta{token},如果 \meta{token} 存在则结果为 \cs{BooleanTrue},
%     否则为 \cs{BooleanFalse}。给定为 \texttt{t}\meta{token}。
%   \item[e] 给定为 \texttt{e}\marg{tokens},一组可选的\emph{修饰},
%     每个修饰都需要一个\emph{值}。如果修饰不存在,则返回 |-NoValue-|。
%     每个修饰给出一个参数,顺序与参数规范中 \meta{tokens} 列表的顺序相同。
%     所有 \meta{tokens} 必须是不同的。\emph{这是一个实验性质的类型}。
%   \item[E] 与 \texttt{e} 相同,但如果没有给出值,则返回一个或多个 \meta{defaults}:
%     \texttt{E}\marg{tokens} \marg{defaults}。详见第~\ref{sec:embellishment} 节。
% \end{itemize}
%
% 利用这些说明符,可以非常容易地创建复杂的输入语法。例如,给定参数定义
% `|s o o m O{default}|',输入 `|*[Foo]{Bar}|' 将被解析为:
% \begin{itemize}[nolistsep]
%   \item |#1| = |\BooleanTrue|
%   \item |#2| = |Foo|
%   \item |#3| = |-NoValue-|
%   \item |#4| = |Bar|
%   \item |#5| = |default|
% \end{itemize}
% 而 `|[One][Two]{}[Three]|' 将被解析为:
% \begin{itemize}[nolistsep]
%   \item |#1| = |\BooleanFalse|
%   \item |#2| = |One|
%   \item |#3| = |Two|
%   \item |#4| = ||
%   \item |#5| = |Three|
% \end{itemize}
%
% 分隔参数类型(\texttt{d}、\texttt{o} 和 \texttt{r})被定义为在收集参数时需要匹配的分隔符对。
% 例如:
% \begin{verbatim}
%   \NewDocumentCommand{\foo}{o}{#1}
%   \foo[[content]] % #1 = "[content]"
%   \foo[[]         % 错误:缺少闭合的 "]"
% \end{verbatim}
% 还要注意,|{| 和 |}| 不能用作分隔符,因为它们被 \TeX{} 用作分组标记。
% 隐式的开始或结束组标记(比如 |\bgroup| 和 |\egroup|)不允许用于分隔参数类型。
% 在这些标记内部抓取的参数必须创建为 \texttt{m} 或 \texttt{g} 类型参数。
%
% 在分隔参数内部,非平衡或其他尴尬的标记可以通过用花括号保护整个参数来包含:
% \begin{verbatim}
%   \NewDocumentCommand{\foobar}{o}{#1}
%   \foobar[{[}]         % 允许,因为 "[" 被 '隐藏'
% \end{verbatim}
% 这些花括号只有在围绕可选参数的\emph{整个}内容时才会被剥离:
% \begin{verbatim}
%   \NewDocumentCommand{\foobaz}{o}{#1}
%   \foobaz[{abc}]         % => "abc"
%   \foobaz[ {abc}]         % => " {abc}"
% \end{verbatim}
%
% 当创建参数说明符时,另外两个字符具有特殊含义。首先,\texttt{+} 用于将参数设为长参数
% (接受段落标记)。与 \LaTeXe{} 的 \cs{newcommand} 不同的是,这是基于每个参数的设定。
% 因此,将示例修改为 `|s o o +m O{default}|' 意味着强制参数现在是 \cs{long},
% 而可选参数则不是。
%
% 第二,字符 \texttt{>} 用于声明所谓的\enquote{参数处理器},可用于修改参数内容,然后将其传递给宏定义。
% 参数处理器的使用是一个稍微高级的主题(或者至少是一个不太常用的功能),在第~\ref{sec:processors} 节进行了介绍。
%
% 当可选参数后跟着具有相同分隔符的强制参数时,\pkg{xparse} 会发出警告,因为用户无法省略
% 可选参数,实际上变成了强制参数。这适用于 \texttt{o}、\texttt{d}、\texttt{O}、\texttt{D}、
% \texttt{s}、\texttt{t}、\texttt{e} 和 \texttt{E} 类型的参数,后面跟着 \texttt{r} 或 \texttt{R} 类型
% 的必选参数,还适用于 \texttt{g} 或 \texttt{G} 类型的参数,后面跟着 \texttt{m} 类型参数。
%
% 由于 \pkg{xparse} 还用于描述出现在更广泛的 \LaTeXe{} 生态系统中的接口,它还定义了额外的参数类型,
% 在第~\ref{sec:backwards} 节描述:强制类型 \texttt{l} 和 \texttt{u},以及可选大括号组类型
% \texttt{g} 和 \texttt{G}。不建议使用它们,因为如果所有包都使用类似的语法,对用户来说会更简单。
% 出于同样的原因,分隔参数 \texttt{r}、\texttt{R}、\texttt{d} 和 \texttt{D} 通常应使用自然配对的
% 分隔符,如 |[| 和 |]| 或 |(| 和 |)|,或者使用相同的分隔符,比如 |"| 和 |"|。一个非常常见的语法是,
% 一个可选参数 \texttt{o} 被视为键值对列表(例如使用 \pkg{l3keys}),
% 后面跟着一些强制参数 \texttt{m}(或 \texttt{+m})。
%
% \subsection{间距和可选参数}
%
% \TeX{} 将会找到函数名后的第一个参数,而不管中间有多少空格。这对于强制参数和
% 可选参数都适用。因此,|\foo[arg]| 和 \verb*|\foo   [arg]| 是等价的。当收集参数
% 到最后一个必须收集的强制参数时(因为它必须存在),在收集参数期间,空格也会被忽略。
%
% 因此,在
% \begin{verbatim}
%   \NewDocumentCommand \foo { m o m } { ... }
% \end{verbatim}
% 用户输入 |\foo{arg1}[arg2]{arg3}| 和 \verb*|\foo{arg1}  [arg2]   {arg3}| 将以
% 相同的方式解析。
%
% 可选参数在任何必选参数之后的行为是可选择的。标准设置允许在此处使用空格,因此在
% \begin{verbatim}
%   \NewDocumentCommand \foobar { m o } { ... }
% \end{verbatim}
% |\foobar{arg1}[arg2]| 和 \verb*|\foobar{arg1} [arg2]| 都会找到一个可选参数。
%
% 这可以通过在参数规范中加入修改后的 |!| 来改变:
% \begin{verbatim}
%   \NewDocumentCommand \foobar { m !o } { ... }
% \end{verbatim}
% 在这种情况下,\verb*|\foobar{arg1} [arg2]| 将不会找到一个可选参数。
%
% 这里有一个微妙之处,是由于 \TeX{} 对\enquote{控制符号}的处理不同,
% 其中命令名由单个字符组成,比如\enquote{\cmd{\\}}。在这种情况下,
% \TeX{} 不会忽略空格,因此可以要求一个可选参数直接跟在这样的命令后面。
% 最常见的例子是在 \pkg{amsmath} 环境中使用 \cmd{\\}。在 \pkg{xparse}
% 中,其签名为
% \begin{verbatim}
%   \DeclareDocumentCommand \\ { !s !o } { ... }
% \end{verbatim}
%
% \subsection{必须分隔的参数}
%
% 分隔参数(\texttt{D} 类型)和\enquote{必需分隔}参数(\texttt{R} 类型)
% 之间的区别在于,如果缺少后者,将引发错误。因此,例如
% \begin{verbatim}
%   \NewDocumentCommand {\foobaz} {r()m} {}
%   \foobaz{oops}
% \end{verbatim}
% 将导致发出错误消息。插入标记 |-NoValue-|(\texttt{r} 类型)或用户
% 指定的默认值(对于 \texttt{R} 类型)以允许错误恢复。
%
% \subsection{Verbatim参数}
%
% 类型为 \texttt{v} 的参数以verbatim模式读取,这将导致所抓取的参数由类别码为
% $12$(\enquote{其它})和 $13$(\enquote{活动})的记号组成,除了空格外,空格
% 被赋予类别码 $10$(\enquote{空格})。参数的定界方式类似于 \LaTeXe{} 的
% \cs{verb} 函数,或者由(正确嵌套的)大括号对定界。
%
% 包含verbatim参数的函数不能出现在其他函数的参数中。 \texttt{v} 参数说明符包含代码
% 来检查这一点,如果所抓取的参数已被 \TeX{} 以不可逆的方式记号化,它将引发
% 错误。
%
% 默认情况下,\texttt{v} 类型的参数最多只能有一行。在前面加上 \texttt{+} 可
% 以允许参数内换行。
%
% 用户应注意,对于verbatim参数的支持在某种程度上是实验性的。因此,在 \texttt{LaTeX-L}
% 邮件列表上非常欢迎反馈意见。
%
% \subsection{参数的默认值}
% \label{sec:defaultvaluesofarguments}
%
% 大写的参数类型(\texttt{O},\texttt{D},\ldots{})允许指定参数缺失时使用的默认值;
% 其小写对应项使用特殊标记 |-NoValue-|。默认值可以用 |#1|,|#2| 等参数的值来表达。
% \begin{verbatim}
%   \NewDocumentCommand {\conjugate} { m O{#1ed} O{#2} } {(#1,#2,#3)}
%   \conjugate {walk}            % => (walk,walked,walked)
%   \conjugate {find} [found]    % => (find,found,found)
%   \conjugate {do} [did] [done] % => (do,did,done)
% \end{verbatim}
% 默认值可以引用参数规范中后面出现的参数。例如,一个命令可以接受两个默认相等的可选参数:
% \begin{verbatim}
%   \NewDocumentCommand {\margins} { O{#3} m O{#1} m } {(#1,#2,#3,#4)}
%   \margins {a} {b}              % => {(-NoValue-,a,-NoValue-,b)}
%   \margins [1cm] {a} {b}        % => {(1cm,a,1cm,b)}
%   \margins {a} [1cm] {b}        % => {(1cm,a,1cm,b)}
%   \margins [1cm] {a} [2cm] {b}  % => {(1cm,a,2cm,b)}
% \end{verbatim}
%
% 用户应注意,对于默认参数引用其他参数的支持在某种程度上是实验性的。因此,在
% \texttt{LaTeX-L} 邮件列表上非常欢迎反馈意见。
%
% \subsection{对于\enquote{修饰符}的默认值}
% \label{sec:embellishment}
%
% \texttt{E} 类型的参数允许每个测试记号有一个默认值。这是通过为列表中的每个
% 条目提供一个默认值列表来实现的,例如:
% \begin{verbatim}
%   E{^_}{{UP}{DOWN}}
% \end{verbatim}
% 如果默认值列表比测试记号列表\emph{短},则会返回特殊的 |-NoValue-| 标记
% (与 \texttt{e} 类型参数一样)。因此,例如
% \begin{verbatim}
%   E{^_}{{UP}}
% \end{verbatim}
% 对于 |^| 测试字符具有默认值 \texttt{UP},但对于 |_| 将返回 |-NoValue-|
% 标记作为默认值。这允许混合显式默认值与检测缺失值。
%
% \subsection{环境的主体}
% \label{sec:body}
%
% 虽然 |\begin|\marg{environment} \dots{}
%   |\end|\marg{environment} 环境通常用于实现\hfil\break \meta{environment} 的代码不需要访问环境
%   内容(其\enquote{主体})的情况,但有时将主体作为标准参数会很有用。
%
% 在 \pkg{xparse} 中,通过在参数规范末尾加上~\texttt{b} 来实现这一点。在 \pkg{xparse}
% 中采取的方法与早期的包 \pkg{environ} 或 \pkg{newenviron} 不同:环境的主体作为一个普
% 通的参数 |#1|、|#2| 等传递给代码部分,而不是存储在诸如 \cs[no-index]{BODY} 这样的
% 宏中。
%
% 例如
% \begin{verbatim}
%   \NewDocumentEnvironment { twice }
%     { O{\ttfamily} +b }
%     {#2#1#2} {}
%   \begin{twice}[\itshape]
%     Hello world!
%   \end{twice}
% \end{verbatim}
% 排版出 \enquote{Hello world!{\itshape Hello world!}}。
%
% 前缀 |+| 用于允许环境主体中有多个段落。也可以对 \texttt{b} 参数应用参数处理器。
%
% 默认情况下,在主体两端修剪空格:否则在 |[\itshape]| 和 |world!| 后面的行尾会有空格。
% 在 \texttt{b} 前放置前缀 |!| 可以抑制修剪空格。
%
% 当参数规范中使用 \texttt{b} 时,\cs{NewDocumentEnvironment} 的最后一个参数,
% 即在 |\end|\marg{environment} 处插入的\meta{end code},是多余的,因为可以简单
% 地将该代码放在\meta{start code} 的末尾。尽管如此,这个(空的)\meta{end code} 必须提供。
%
% 使用此功能的环境可以嵌套。
%
% 用户应注意,此功能在某种程度上是实验性的。因此,在 \texttt{LaTeX-L} 邮件列表上非常
% 欢迎反馈意见。
%
% \subsection{带星号的环境}
%
% 许多宏包用于定义带有和不带有 \texttt{*} 的环境,例如 \texttt{tabular} 和 \texttt{tabular*}。
% 目前,\pkg{xparse} 并没有提供专门的工具来定义这些带星号的环境:应该简单地分别定义这两种
% 环境,例如:
% \begin{verbatim}
% \NewDocumentEnvironment { tabular } { o +m } {...} {...}
% \NewDocumentEnvironment { tabular* } { m o +m } {...} {...}
% \end{verbatim}
% 当然,在这个例子中标为 \enquote{\texttt{...}} 的这两种环境的实现可以依赖于相同的内部命令。
%
% 请注意,这种情况与 \texttt{s} 类型的参数不同:如果环境的签名以~\texttt{s} 开头,那么在
% \cs{begin} 的参数之后会搜索星号。例如,以下示例会排版出 \texttt{star}。
% \begin{verbatim}
% \NewDocumentEnvironment { envstar } { s }
%   {\IfBooleanTF {#1} {star} {no star}} {}
% \begin{envstar}*
% \end{envstar}
% \end{verbatim}
%
% \subsection{向前兼容性}
% \label{sec:backwards}
%
% \pkg{xparse} 的一个作用是描述现有的 \LaTeX{} 接口,其中包括一些在 \LaTeX{} 中相当不寻常
% (与 plain \TeX{} 等格式相比)的接口,比如分界符参数。因此,该宏包定义了一些参数说明符,目前
% 在编写宏包时应尽量避免使用,因为使用它们会导致用户接口不一致。最简单的语法通常是最好的,
% 使用参数说明符如 |mmmm| 或 |ommmm|,即一个可选参数后跟一些标准的必选参数。可以使用 \pkg{l3keys}
% 提供的工具使可选参数支持键-值语法。
%
% 不再推荐使用的参数类型包括:
% \begin{itemize}[font=\ttfamily]
%   \item[l] 读取直到第一个开始组标记之前的所有内容的必选参数:在标准 \LaTeX{} 中,这是左花括号。
%   \item[u] 读取必选参数 \enquote{直到} 遇到 \meta{tokens},其中期望的 \meta{tokens} 作为该说明符的
%     参数给出:\texttt{u}\marg{tokens}。
%   \item[g] 给定在一对 \TeX{} 组标记内部的可选参数(在标准 \LaTeX{} 中为 |{| \ldots |}|),
%     如果不存在则返回 |-NoValue-|。
%   \item[G] 类似于 \texttt{g},但如果没有给定值则返回 \meta{default}:\texttt{G}\marg{default}。
% \end{itemize}
%
% \subsection{关于参数分界符的细节}
%
% 在普通(不可展开)命令中,分界符类型会通过向前查看(使用 \pkg{expl3} 的 |\peek_...| 函数)来寻找初始分界符。
% 分界符标记必须具有与定义为分界符的标记相同的含义和“形状”。分界符有三种可能的情况:字符标记、控制序列标记和活动字符标记。
% 在本描述的所有实际情况中,活动字符标记将与控制序列标记的行为完全相同。
%
% \subsubsection{字符标记}
%
% 字符标记由其字符代码和其含义(类别码 |\catcode|)所特征化。当定义命令时,字符标记的含义将固定在命令的定义中,不可更改。
% 如果在定义时打开分界符具有与定义时相同的字符和类别码,则命令将正确地将其视为参数分界符。
% 例如,在以下代码中:
% \begin{verbatim}
%   \NewDocumentCommand { \foobar } { D<>{default} } {(#1)}
%   \foobar <hello> \par
%   \char_set_catcode_letter:N <
%   \foobar <hello>
% \end{verbatim}
% 输出将是:
% \begin{verbatim}
%   (hello)
%   (default)<hello>
% \end{verbatim}
% 因为在两次调用 |\foobar| 之间,打开分界符 |<| 的含义发生了变化,所以第二次调用不会将 |<| 视为有效分界符。
% 命令假设如果找到了有效的打开分界符,就会有一个匹配的关闭分界符。如果没有(要么是被省略了,要么是发生了含义变化),
% 就会引发低级 \TeX{} 错误并中止命令调用。
%
% \subsubsection{控制序列标记}
%
% 控制序列(或控制字符)标记由其名称所特征化,其含义是其定义。
% 一个标记不能同时具有两个不同的含义。
% 当控制序列被定义为命令中的分界符时,无论其当前定义如何,只要在文档中找到该控制序列名称,就会将其检测为分界符。
% 例如,在以下代码中:
% \begin{verbatim}
%   \cs_set:Npn \x { abc }
%   \NewDocumentCommand { \foobar } { D\x\y{default} } {(#1)}
%   \foobar \x hello\y \par
%   \cs_set:Npn \x { def }
%   \foobar \x hello\y
% \end{verbatim}
% 输出将是:
% \begin{verbatim}
%   (hello)
%   (hello)
% \end{verbatim}
% 在两次调用中,都会看到 |\x| 作为分界符。
%
% \section{声明命令和环境}
%
% 有了参数规范的概念,现在可以描述使用 \pkg{xparse} 创建函数和环境的方法。
%
% 接口构建命令是在 \LaTeX3 中创建文档级函数的首选方法。通过这种方式生成的所有
% 函数都是天然健壮的(robust)(使用 \eTeX{} 的 \cs{protected} 机制)。
%
% \begin{function}
%   {
%     \NewDocumentCommand     ,
%     \RenewDocumentCommand   ,
%     \ProvideDocumentCommand ,
%     \DeclareDocumentCommand
%   }
%   \begin{syntax}
%     \cs{NewDocumentCommand} \meta{function} \Arg{arg spec} \Arg{code}
%   \end{syntax}
%   这组命令用于创建文档级别的 \meta{function}。函数的参数规范由 \meta{arg
%   spec} 给出,并且函数通过 \pkg{xparse} 找到的参数用 \meta{code} 展开,
%   其中 |#1|、|#2| 等被替换为参数。
% \end{function}
%
% 例如:
% \begin{verbatim}
%   \NewDocumentCommand \chapter { s o m }
%     {
%       \IfBooleanTF {#1}
%         { \typesetstarchapter {#3} }
%         { \typesetnormalchapter {#2} {#3} }
%     }
% \end{verbatim}
% 这是定义一个 \cs{chapter} 命令的方式,它基本上的行为类似于当前的 \LaTeXe{}
% 命令(除了它在解析 \texttt{*} 时也接受可选参数)。 \cs{typesetnormalchapter}
% 可以测试它的第一个参数是否为 |-NoValue-|,以确定是否存在可选参数。
%
%   \cs{New\ldots}、\cs{Renew\ldots}、\cs{Provide\ldots} 和
%   \cs{Declare\ldots} 版本的区别在于 \meta{function} 是否已经定义了行为。
%   \begin{itemize}
%     \item \cs{NewDocumentCommand} 如果 \meta{function} 已经定义,会发出错误。
%     \item \cs{RenewDocumentCommand} 如果 \meta{function} 没有之前的定义,会发出错误。
%     \item \cs{ProvideDocumentCommand} 只有在尚未给出 \meta{function} 的定义时才创建新定义。
%     \item \cs{DeclareDocumentCommand} 将始终创建新定义,无论同名的 \meta{function} 是否已存在。应谨慎使用。
%   \end{itemize}
%
%   \begin{texnote}
%     与 \LaTeXe{} 的 \cs{newcommand} 等不同,\cs{NewDocumentCommand} 函数系列不会阻止创建以 \cs{end\ldots} 开头的函数。
%   \end{texnote}
%
% \begin{function}
%   {
%     \NewDocumentEnvironment     ,
%     \RenewDocumentEnvironment   ,
%     \ProvideDocumentEnvironment ,
%     \DeclareDocumentEnvironment
%   }
%   \begin{syntax}
%     \cs{NewDocumentEnvironment} \Arg{environment} \Arg{arg spec}
%     ~~\Arg{start code} \Arg{end code}
%   \end{syntax}
%   这些命令与 \cs{NewDocumentCommand} 等类似,但创建环境(\cs{begin}\Arg{environment} \ldots{}
%   \cs{end}\Arg{environment})。 \meta{start code} 和 \meta{end code} 都可以访问由 \meta{arg spec}
%   定义的参数。参数将在 \cs{begin}\Arg{environment} 之后给出。
% \end{function}
%
% \section{其他 \pkg{xparse} 命令}
%
% \subsection{测试特殊值}
%
% 使用 \pkg{xparse} 创建的可选参数利用专用变量返回有关接收到的参数性质的信息。
%
% \begin{function}[EXP]{\IfNoValueT, \IfNoValueF, \IfNoValueTF}
%   \begin{syntax}
%     \cs{IfNoValueTF} \Arg{argument} \Arg{true code} \Arg{false code}
%     \cs{IfNoValueT} \Arg{argument} \Arg{true code}
%     \cs{IfNoValueF} \Arg{argument} \Arg{false code}
%   \end{syntax}
%   \cs{IfNoValue(TF)} 测试用于检查 \meta{argument}(|#1|、|#2| 等)
%   是否是特殊的 |-NoValue-| 标记。例如
%   \begin{verbatim}
%     \NewDocumentCommand \foo { o m }
%       {
%         \IfNoValueTF {#1}
%           { \DoSomethingJustWithMandatoryArgument {#2} }
%           {  \DoSomethingWithBothArguments {#1} {#2}   }
%       }
%   \end{verbatim}
%   如果给定了可选参数,将使用不同的内部函数,而不是当它不存在时。
%
%   请注意,有三个测试可用,取决于所需的结果分支:\cs{IfNoValueTF}、
%   \cs{IfNoValueT} 和 \cs{IfNoValueF}。由于 \cs{IfNoValue(TF)} 测试是
%   可展开的,因此可以稍后测试这些值,例如在排版或扩展上下文中。
%
%   需要注意的是 |-NoValue-| 被构造成\emph{不会}与简单文本输入 |-NoValue-|
%   匹配,也就是说
%   \begin{verbatim}
%     \IfNoValueTF{-NoValue-}
%   \end{verbatim}
%   在逻辑上是 \texttt{false} 的。当两个可选参数依次跟随(这是我们通常不
%   鼓励的语法)时,允许命令的使用者仅提供第二个参数并提供空的第一个参数
%   是有意义的。然而,与其分别测试空值和 |-NoValue-|,最好使用参数类型为 |O|
%   并使用空默认值,并仅使用 \pkg{expl3} 条件 \cs{tl_if_blank:nTF} 或其
%   \pkg{etoolbox} 类似物 \tn{ifblank} 测试是否为空。
% \end{function}
%
% \begin{function}[EXP]{\IfValueT, \IfValueF, \IfValueTF}
%   \begin{syntax}
%     \cs{IfValueTF} \Arg{argument} \Arg{true code} \Arg{false code}
%   \end{syntax}
%   \cs{IfNoValue(TF)} 的相反形式也作为 \cs{IfValue(TF)} 提供。
%   根据特定代码场景,上下文将决定哪种逻辑形式更合理。
% \end{function}
%
% \begin{variable}{\BooleanFalse, \BooleanTrue}
%   在搜索可选字符时(使用 \texttt{s} 或 \texttt{t\meta{char}}),设置的
%   \texttt{true} 和 \texttt{false} 标志具有可以在代码块外部访问的名称。
% \end{variable}
%
% \begin{function}[EXP]{\IfBooleanT, \IfBooleanF, \IfBooleanTF}
%   \begin{syntax}
%     \cs{IfBooleanTF} \Arg{argument} \Arg{true code} \Arg{false code}
%   \end{syntax}
%   用于测试 \meta{argument}(|#1|、|#2| 等)是否为 \cs{BooleanTrue} 或
%   \cs{BooleanFalse}。例如
%   \begin{verbatim}
%     \NewDocumentCommand \foo { s m }
%       {
%         \IfBooleanTF {#1}
%           { \DoSomethingWithStar {#2} }
%           { \DoSomethingWithoutStar {#2} }
%       }
%   \end{verbatim}
%   检查第一个参数是否为星号,然后根据这一信息选择要采取的操作。
% \end{function}
%
% \subsection{参数处理器}
% \label{sec:processors}
%
% \pkg{xparse}引入了参数处理器的概念,它在底层系统抓取参数之后但在传递给
% \meta{code}之前应用。因此,参数处理器可用于在早期规范输入,使内部函数
% 完全独立于输入形式。处理器应用于用户输入和可选参数的默认值,但\emph{不}
% 适用于特殊的 |-NoValue-| 标记。
%
% 每个参数处理器由语法 \texttt{>}\marg{processor} 指定。处理器从右向左
% 应用,因此
% \begin{verbatim}
%   >{\ProcessorB} >{\ProcessorA} m
% \end{verbatim}
% 将对 \texttt{m} 参数抓取的标记应用 \cs{ProcessorA},然后是 \cs{ProcessorB}。
%
% 有时使用另一个参数的值作为处理器的一个参数可能会有用。例如,使用以下
% 定义的 \cs{SplitList} 处理器,
% \begin{verbatim}
%   \NewDocumentCommand \foo { O{,} >{\SplitList{#1}} m } { \foobar{#2} }
%   \foo{a,b;c,d}
% \end{verbatim}
% 结果是 |\foobar| 接收参数 |{a}{b;c}{d}|,因为 \cs{SplitList} 接收两个参数,
% 可选参数(这里的默认值是逗号)和必选参数。总结一下,首先在输入中搜索参数,
% 然后确定任何默认参数,就像第\ref{sec:defaultvaluesofarguments}节中解释的那样,
% 然后将这些默认参数传递给任何处理器。在处理器中引用参数(通过 |#1|、|#2|等)
% 时,使用的始终是任何处理器之前的参数。
%
% \begin{variable}{\ProcessedArgument}
%   \pkg{xparse} 定义了一组非常小的处理器函数。主要预计代码编写者将想要创建
%   自己的处理器。这些处理器需要接受一个参数,即抓取的标记(或前一个处理器
%   函数返回的标记)。处理器函数应将处理后的参数作为变量 \cs{ProcessedArgument}返回。
% \end{variable}
%
% \begin{function}{\ReverseBoolean}
%   \begin{syntax}
%     \cs{ReverseBoolean}
%   \end{syntax}
%   此处理器颠倒了 \cs{BooleanTrue} 和 \cs{BooleanFalse} 的逻辑,因此之前的示例将变成
%   \begin{verbatim}
%     \NewDocumentCommand \foo { > { \ReverseBoolean } s m }
%       {
%         \IfBooleanTF #1
%           { \DoSomethingWithoutStar {#2} }
%           { \DoSomethingWithStar {#2} }
%       }
%   \end{verbatim}
% \end{function}
%
% \begin{function}[updated = 2012-02-12]{\SplitArgument}
%   \begin{syntax}
%     \cs{SplitArgument} \Arg{number} \Arg{token(s)}
%   \end{syntax}
%   此处理器将给定的参数在每个 \meta{tokens} 出现的地方分割,最多分割
%   \meta{number} 个标记(因此将输入分成 $\text{\meta{number}} + 1$ 部分)。
%   如果输入中存在过多的 \meta{tokens},将报错。处理后的输入放置在
%   $\text{\meta{number}} + 1$ 组大括号中供进一步使用。如果参数中
%   \Arg{tokens} 的数量少于 \Arg{number},则在处理后的参数末尾添加
%   |-NoValue-| 标记。
%   \begin{verbatim}
%     \NewDocumentCommand \foo
%       { > { \SplitArgument { 2 } { ; } } m }
%       { \InternalFunctionOfThreeArguments #1 }
%   \end{verbatim}
%   如果分割使用了单个字符 \meta{token},则在分割之前将替换任何匹配
%   \meta{token} 的类别码为 $13$(活动状态)的字符。
%   每个解析的项目的两端都会去除空格。
% \end{function}
%
% \begin{function}{\SplitList}
%   \begin{syntax}
%     \cs{SplitList} \Arg{token(s)}
%   \end{syntax}
%   此处理器将给定的参数在每个 \meta{token(s)} 出现的地方分割,其中
%   项目数目不是固定的。然后在 |#1| 中将每个项目放置在大括号中。结果是,
%   处理后的参数可以使用映射函数进一步处理。
%   \begin{verbatim}
%     \NewDocumentCommand \foo
%       { > { \SplitList { ; } } m }
%       { \MappingFunction #1 }
%   \end{verbatim}
%   如果分割使用了单个字符 \meta{token},则在分割之前将替换任何匹配
%   \meta{token} 的类别码为 $13$(活动状态)的字符。
%   每个解析的项目的两端都会去除空格。
% \end{function}
%
% \begin{function}[EXP]{\ProcessList}
%   \begin{syntax}
%     \cs{ProcessList} \Arg{list} \Arg{function}
%   \end{syntax}
%   为了支持 \cs{SplitList},函数 \cs{ProcessList} 可用于对 \meta{list}
%   中的每个条目应用 \meta{function}。 \meta{function} 应吸收一个参数:列
%   表条目。例如
%   \begin{verbatim}
%     \NewDocumentCommand \foo
%       { > { \SplitList { ; } } m }
%       { \ProcessList {#1} { \SomeDocumentFunction } }
%   \end{verbatim}
%
%   \textbf{此函数为实验性质。}
% \end{function}
%
% \begin{function}{\TrimSpaces}
%   \begin{syntax}
%     \cs{TrimSpaces}
%   \end{syntax}
%   去除参数两端的任何前导和尾随空格(字符码为 $32$,类别码为 $10$ 的
%   标记)。因此,例如声明一个函数
%   \begin{verbatim}
%     \NewDocumentCommand \foo
%       { > { \TrimSpaces } m }
%       { \showtokens {#1} }
%   \end{verbatim}
%   并在文档中使用它,如
%   \begin{verbatim}
%     \foo{ hello world }
%   \end{verbatim}
%   将在终端显示 \texttt{hello world},两端的空格已被移除。
%   \cs{TrimSpaces} 将删除多个空格,若它们被包含进来,以至于标准 \TeX{}
%   将多个空格转换为单个空格的规则不适用。
%
%   \textbf{此函数为实验性质。}
% \end{function}
%
% \subsection{可完全展开的文档命令}
%
% 在极为罕见的情况下,可能有用的是使用可完全展开的参数抓取器来创建函
% 数。为了支持这一点,\pkg{xparse}可以创建可展开函数以及通常的健壮函
% 数。这给函数所接受的参数的性质以及它所实现的代码施加了许多限制。这个
% 功能只应在\emph{绝对必要}时使用;如果你不了解何时可能需要使用它们,
% \emph{请不要使用这些函数}!
%
% \begin{function}
%   {
%     \NewExpandableDocumentCommand     ,
%     \RenewExpandableDocumentCommand   ,
%     \ProvideExpandableDocumentCommand ,
%     \DeclareExpandableDocumentCommand
%   }
%   \begin{syntax}
%     \cs{NewExpandableDocumentCommand}
%     ~~~~\meta{function} \Arg{arg spec} \Arg{code}
%   \end{syntax}
%   这组命令用于创建一个文档级别的 \meta{function},它将以可完全展开
%   的方式抓取其参数。该函数的参数规范由 \meta{arg spec} 给出,并且函
%   数将执行 \meta{code}。通常,\meta{code} 也将是可完全展开的,尽管
%   也可能不是这种情况(例如,用于表格的函数可能会展开,以便 \cs{omit}
%   是第一个非可展开的非空格标记)。
%
%   可展开地解析参数施加了许多限制,涉及到可以读取的参数类型以及可用的
%   错误检查:
%   \begin{itemize}
%     \item 如果存在最后一个参数,则必须是 \texttt{m}、\texttt{r}、
%       \texttt{R}、\texttt{l} 或 \texttt{u} 中的一个。
%     \item 所有短参数都出现在长参数之前。
%     \item 可选参数后面不能使用必需参数类型 \texttt{l} 和 \texttt{u}。
%     \item 不可用可选参数类型 \texttt{g} 和 \texttt{G}。
%     \item 不可用的 \enquote{verbatim} 参数类型 \texttt{v}。
%     \item 不可用参数处理器(使用 \texttt{>})。
%     \item 不可能区分,例如 |\foo[| 和 |\foo{[}|:在这两种情况下,
%       \texttt{[} 都将被解释为可选参数的开始。因此,对可选参数的检查
%       不够健壮,比标准版本差。
%   \end{itemize}
%   如果给出的参数说明符不符合前六个要求,\pkg{xparse} 将会发出错误。
%   最后一条是在函数使用时出现的问题,因此超出了 \pkg{xparse} 本身的范
%   围。
% \end{function}
%
% \subsection{获取参数规范}
%
% 文档命令和环境的参数规范可供检查和使用。
%
% \begin{function}{\GetDocumentCommandArgSpec, \GetDocumentEnvironmentArgSpec}
%   \begin{syntax}
%     \cs{GetDocumentCommandArgSpec} \meta{function}
%     \cs{GetDocumentEnvironmentArgSpec} \Arg{environment}
%   \end{syntax}
%   这些函数将请求的 \meta{function} 或 \meta{environment} 的当前参数规范转移
%   到记号列表变量 \cs{ArgumentSpecification} 中。如果 \meta{function} 或
%   \meta{environment} 没有已知的参数规范,则会发出错误。对 \cs{ArgumentSpecification}
%   的赋值局限于当前的 \TeX{} 组。
% \end{function}
%
% \begin{function}
%   {\ShowDocumentCommandArgSpec,  \ShowDocumentEnvironmentArgSpec}
%   \begin{syntax}
%     \cs{ShowDocumentCommandArgSpec} \meta{function}
%     \cs{ShowDocumentEnvironmentArgSpec} \Arg{environment}
%   \end{syntax}
%   这些函数在终端显示请求的 \meta{function} 或 \meta{environment} 的当前参数规范。
%   如果 \meta{function} 或 \meta{environment} 没有已知的参数规范,则会发出错误。
% \end{function}
%
% \section{加载时选项}
%
% \DescribeOption{log-declarations}
% 该包识别加载时选项 \texttt{log-declarations},这是一个键-值选项,
% 接受值 \texttt{true} 和 \texttt{false}。默认情况下,该选项设置为
% \texttt{false},意味着不记录任何声明的命令或环境。通过使用以下方式加载
% \pkg{xparse}:
% \begin{verbatim}
%   \usepackage[log-declarations=true]{xparse}
% \end{verbatim}
% 每个新声明的命令或环境都将被记录。
%
% \end{documentation}
%
% \begin{implementation}
%
% \section{\pkg{xparse}实现}
%
%    \begin{macrocode}
%<*2ekernel|package>
%    \end{macrocode}
%
%    \begin{macrocode}
%<@@=cmd>
%    \end{macrocode}
%
% 该包文件与 \pkg{xparse-2018-04-12}、\pkg{xparse-2020-10-01} 和
% |xparse-generic.tex| 中的冻结版本旨在跨不同的 \LaTeX{} 发布版本中工作,
% 符合 \pkg{xparse} 所需的最小 \pkg{expl3} 版本。
%
% 我们无法在这里使用 \cs{DeclareRelease} 的 \pkg{latexrelease} 机制,
% 因为前进或后退的情况不同,由于从 |xparse| 到 |cmd| 的前缀更改,所以
% 我们自行检查以确保加载了正确的版本。
%
% 所有这些加载假设,如果应加载 \pkg{latexrelease},那么它应该在 \pkg{xparse}
% 之前加载。反之则完全没有进行测试,很可能会导致故障。
%
% 在 2020-10-01 之前的发布版中,\cs{NewDocumentCommand} 未定义,因此未
% 加载 \pkg{xparse},因此我们可以从 \pkg{xparse-2018-04-12} 加载完整版本。
% 否则,我们将仅从 \pkg{xparse-2020-10-01} 加载已弃用的参数类型。如果我们
% 处于 |2ekernel| 模式,则意味着向前滚动,因此仅加载来自 |xparse-generic.tex|
% 的代码。
%
% 在 |2ekernel| 模式中,我们预期定义了两个宏来解析日期,以便我们可以正确比较
% \cs{fmtdate}。
%    \begin{macrocode}
%<*2ekernel>
\def\@parse@version#1/#2/#3#4#5\@nil{%
  \@parse@version@dash#1-#2-#3#4\@nil}
\def\@parse@version@dash#1-#2-#3#4#5\@nil{%
  \if\relax#2\relax\else#1\fi#2#3#4 }
%</2ekernel>
\ExplSyntaxOn
\cs_set_protected:Npn \@@_tmp:w #1
  {
    \DeclareOption* { \PassOptionsToPackage { \CurrentOption } {#1} }
    \ProcessOptions \relax
    \RequirePackage {#1}
  }
\cs_if_free:NTF \NewDocumentCommand
  {
    \ExplSyntaxOff
    \ifnum\expandafter
        \@parse@version\fmtversion//00\@nil <
        \@parse@version 2020-10-01//00\@nil
      \@@_tmp:w { xparse-2018-04-12 }
    \else
%<2ekernel>      \@@@@input xparse-generic.tex ~
%<package>      \@@_tmp:w { xparse-2020-10-01 }
    \fi
    \file_input_stop:
  }
%    \end{macrocode}
%
% 如果 \cs{NewDocumentCommand} 已经定义,我们要么处于 \LaTeX{} 2020-10-01
% 或更新版本。在前一种情况下,内核中加载的代码具有 |__xparse| 前缀,因此我们
% 将加载 \pkg{xparse-2020-10-01};否则,我们将继续使用 |xparse.sty|,其中
% 包含带有 |__cmd| 前缀的 \pkg{xparse} 的最后残余部分。要检查这一点,我们只需
% 查看一个带有 |__cmd| 前缀的内部命令。
%    \begin{macrocode}
  {
    \ExplSyntaxOff
    \cs_if_exist:NF \@@_start:nNNnnn
      {
        \@@_tmp:w { xparse-2020-10-01 }
        \file_input_stop:
      }
  }
\ExplSyntaxOff
%    \end{macrocode}
%
%    \begin{macrocode}
%</2ekernel|package>
%    \end{macrocode}
%
% 在旧版本中,前缀是 |xparse|,但自 \LaTeXe{} 2021 春季版本起,核心代码已
% 包含在内核中,而此文件仅保存了不推荐使用的参数说明符 |G|、|l| 和 |u|。为了
% 使 |xparse.sty| 加载时不推荐使用的类型能够正常工作,匹配了 \LaTeXe{} 内核
% 中的前缀,因此前缀已更改为 |cmd|。
%
%    \begin{macrocode}
%<*package>
%    \end{macrocode}
%
%    \begin{macrocode}
\ProvidesExplPackage{xparse}{2023-10-10}{}
  {L3 Experimental document command parser}
%    \end{macrocode}
%
% \subsection{包选项}
%
% \begin{variable}{\l_@@_options_clist}
% \begin{variable}{\l_@@_log_bool}
% 用键-值选项手动记录信息:手动操作以降低依赖性。
%    \begin{macrocode}
\clist_new:N \l_@@_options_clist
\DeclareOption* { \clist_put_right:NV \l_@@_options_clist \CurrentOption }
\ProcessOptions \relax
\cs_set_protected:Npn \@@_tmp:w #1
  {
    \keys_define:nn {#1}
      {
        log-declarations .bool_set:N = \l_@@_log_bool ,
        log-declarations .initial:n  = false
      }
    \keys_set:nV {#1} \l_@@_options_clist
    \bool_if:NTF \l_@@_log_bool
      { \msg_redirect_module:nnn {#1} { info } { } }
      { \msg_redirect_module:nnn {#1} { info } { none } }
    \cs_new_protected:Npn \@@_unknown_argument_type_error:n ##1
      {
        \msg_error:nnee {#1} { unknown-argument-type }
          { \@@_environment_or_command: } { \tl_to_str:n {##1} }
      }
  }
\msg_if_exist:nnTF { cmd } { define-command }
  { \@@_tmp:w { cmd } }
  { \@@_tmp:w { ltcmd } }
%    \end{macrocode}
% \end{variable}
% \end{variable}
%
% \subsection{规范化参数规范}
%
% \begin{macro}{\@@_normalize_arg_spec_loop:n}
%   遍历参数规范,调用针对每种参数类型的辅助程序。如果任何参数未知,则停止定义。
%    \begin{macrocode}
\cs_gset_protected:Npn \@@_normalize_arg_spec_loop:n #1
  {
    \quark_if_recursion_tail_stop:n {#1}
    \int_incr:N \l_@@_current_arg_int
    \cs_if_exist_use:cF { @@_normalize_type_ \tl_to_str:n {#1} :w }
      {
        \@@_unknown_argument_type_error:n {#1}
        \@@_bad_def:wn
      }
  }
%    \end{macrocode}
% \end{macro}
%
% \begin{macro}{\@@_normalize_type_g:w}
%   这些参数类型是更一般类型的别名,例如具有默认参数 |-NoValue-|。
%    \begin{macrocode}
\cs_new_protected:Npe \@@_normalize_type_g:w
  { \exp_not:N \@@_normalize_type_G:w { \exp_not:V \c_novalue_tl } }
%    \end{macrocode}
% \end{macro}
%
% \begin{macro}{\@@_normalize_type_G:w}
%   可选参数类型。检查所有必需的数据是否存在(如果适用,是否由单个字符组成),
%   并检查可展开命令的禁用类型。然后,在每种情况下将数据存储在 \cs{l_@@_arg_spec_tl} 中,
%   以供后续检查,在 \cs{l_@@_last_delimiters_tl} 中存储决定是否有可选参数的标记
%   (对于大括号,存储为 |{}|,稍后将其视为空分隔符)。
%    \begin{macrocode}
\cs_new_protected:Npn \@@_normalize_type_G:w #1
  {
    \quark_if_recursion_tail_stop_do:nn {#1} { \@@_bad_arg_spec:wn }
    \@@_normalize_check_gv:N G
    \@@_add_arg_spec:n { G {#1} }
    \tl_put_right:Nn \l_@@_last_delimiters_tl { { } }
    \@@_normalize_arg_spec_loop:n
  }
%    \end{macrocode}
% \end{macro}
%
% \begin{macro}{\@@_normalize_type_l:w,\@@_normalize_type_u:w}
%   必选参数。首先检查所需的数据是否存在,是否由单个字符组成(如果适用),
%   并且如有必要,检查参数类型是否允许用于可展开命令。然后将数据保存在
%   \cs{l_@@_arg_spec_tl} 中,计算必选参数的数量,并清空最后分隔符的列表。
%    \begin{macrocode}
\cs_new_protected:Npn \@@_normalize_type_l:w
  {
    \@@_normalize_check_lu:N l
    \@@_add_arg_spec_mandatory:n { l }
    \@@_normalize_arg_spec_loop:n
  }
\cs_new_protected:Npn \@@_normalize_type_u:w #1
  {
    \quark_if_recursion_tail_stop_do:nn {#1} { \@@_bad_arg_spec:wn }
    \@@_normalize_check_lu:N u
    \@@_add_arg_spec_mandatory:n { u {#1} }
    \@@_normalize_arg_spec_loop:n
  }
%    \end{macrocode}
% \end{macro}
%
% \subsection{设置标准签名}
%
% \begin{macro}{\@@_add_type_G:w}
%   对于 \texttt{G} 类型,抓取器和默认值被添加到签名中。
%    \begin{macrocode}
\cs_new_protected:Npn \@@_add_type_G:w #1
  {
    \@@_flush_m_args:
    \@@_add_default:n {#1}
    \@@_add_grabber:N G
    \@@_prepare_signature:N
  }
%    \end{macrocode}
% \end{macro}
%
% \begin{macro}{\@@_add_type_l:w}
%   寻找 \texttt{l} 参数非常简单:除了添加抓取器外,没有其他要做的事情。
%    \begin{macrocode}
\cs_new_protected:Npn \@@_add_type_l:w
  {
    \@@_flush_m_args:
    \@@_add_default:
    \@@_add_grabber:N l
    \@@_prepare_signature:N
  }
%    \end{macrocode}
% \end{macro}
%
% \begin{macro}{\@@_add_type_u:w}
%   在设置阶段,\texttt{u} 类型参数与 \texttt{G} 类型相同,除了抓取器函数的名称不同。
%    \begin{macrocode}
\cs_new_protected:Npn \@@_add_type_u:w #1
  {
    \@@_flush_m_args:
    \@@_add_default:
    \@@_add_grabber:N u
    \tl_put_right:Nn \l_@@_signature_tl { {#1} }
    \@@_prepare_signature:N
  }
%    \end{macrocode}
% \end{macro}
%
% \subsection{设置可扩展类型}
%
% \begin{macro}{\@@_add_expandable_type_l:w}
%   通过重复使用类型 \texttt{u},因为在 \TeX{} 宏的参数文本以 |#| 结束,
%   实际上就会以开放花括号为定界符。
%    \begin{macrocode}
\cs_new_protected:Npn \@@_add_expandable_type_l:w
  { \@@_add_expandable_type_u:w ## }
%    \end{macrocode}
% \end{macro}
%
% \begin{macro}{\@@_add_expandable_type_u:w}
%   定义一个辅助函数,它将直接用在签名中。它抓取一个由 |#1| 定界的参数,
%   并将其放置在 \cs{q_@@} 之前。
%    \begin{macrocode}
\cs_new_protected:Npn \@@_add_expandable_type_u:w #1
  {
    \@@_add_default:
    \bool_if:NTF \l_@@_long_bool
      { \cs_set:cpn }
      { \cs_set_nopar:cpn }
      { \l_@@_expandable_aux_name_tl } ##1 \q_@@ ##2 ##3 ##4 #1
      { ##1 {##4} \q_@@ ##2 ##3 }
    \@@_add_expandable_grabber:nn { u }
      { \exp_not:c  { \l_@@_expandable_aux_name_tl } }
    \@@_prepare_signature:N
  }
%    \end{macrocode}
% \end{macro}
%
% \subsection{复制命令及其内部结构}
%
% 复制命令的装置几乎完全位于 |ltcmd.dtx| 中,预先加载在 \LaTeXe{} 内核中。
% 关于复制已弃用的参数类型 |G|、|l| 和 |u| 的缺失部分归结为复制 |u| 的可展开抓取器。
%
% \begin{macro}{\@@_copy_grabber_u:w}
%   可展开的 |u| 类型使用专用抓取器,就像 |D| 类型一样,只是它的两个分隔符标记被省略,
%   因此为了复制它,我们只需复制一个 |D| 类型,并将最后两个参数留空:
%    \begin{macrocode}
\cs_new_protected:Npn \@@_copy_grabber_u:w #1 #2 #3
  { \@@_copy_grabber_D:w {#1} {#2} {#3} { } { } }
%    \end{macrocode}
% \end{macro}
%
% \subsubsection{显示命令的定义}
%
% \begin{macro}{\c_@@_show_type_u_tl,\c_@@_show_type_G_tl}
%   与复制相同,此处仅缺少一些小部分。
%   具体来说,有两个令牌列表告诉 \pkg{ltcmd} 机制如何处理这里定义的参数类型。
%   |G| 和 |u| 都被分类为 |3|:带有默认值的命令。对于 |u| 来说实际并非如此,
%   但这足够接近以获得我们想要的输出。
%    \begin{macrocode}
\tl_const:Nn \c_@@_show_type_u_tl { 3 }
\tl_const:Nn \c_@@_show_type_G_tl { 3 }
%    \end{macrocode}
% \end{macro}
%
% \subsection{抓取参数}
%
% \begin{macro}{\@@_grab_G:w}
% \begin{macro}{\@@_grab_G_long:w}
% \begin{macro}{\@@_grab_G_obey_spaces:w}
% \begin{macro}{\@@_grab_G_long_obey_spaces:w}
% \begin{macro}{\@@_grab_G_aux:nNN}
%   可选组通过含义进行检查,以便相同的代码适用于例如类似 \ConTeXt{} 的输入方式。
%    \begin{macrocode}
\cs_new_protected:Npn \@@_grab_G:w #1 \@@_run_code:
  {
    \@@_grab_G_aux:nNN {#1} \cs_set_protected_nopar:Npn
      \@@_peek_nonspace:NTF
  }
\cs_new_protected:Npn \@@_grab_G_long:w #1 \@@_run_code:
  {
    \@@_grab_G_aux:nNN {#1} \cs_set_protected:Npn
      \@@_peek_nonspace:NTF
  }
\cs_new_protected:Npn \@@_grab_G_obey_spaces:w #1 \@@_run_code:
  {
    \@@_grab_G_aux:nNN {#1} \cs_set_protected_nopar:Npn
      \peek_meaning:NTF
  }
\cs_new_protected:Npn \@@_grab_G_long_obey_spaces:w #1 \@@_run_code:
  {
    \@@_grab_G_aux:nNN {#1} \cs_set_protected:Npn
      \peek_meaning:NTF
  }
\cs_new_protected:Npn \@@_grab_G_aux:nNN #1#2#3
  {
    \tl_set:Nn \l_@@_signature_tl {#1}
    \exp_after:wN #2 \l_@@_fn_tl ##1
      { \@@_add_arg:n {##1} }
    #3 \c_group_begin_token
      { \l_@@_fn_tl }
      { \@@_add_arg:o \c_novalue_tl }
  }
%    \end{macrocode}
% \end{macro}
% \end{macro}
% \end{macro}
% \end{macro}
% \end{macro}
%
% \begin{macro}{\@@_grab_l:w}
% \begin{macro}{\@@_grab_l_long:w}
% \begin{macro}{\@@_grab_l_aux:nN}
% 参数抓取器用于必选的 \TeX{} 参数非常简单。
%    \begin{macrocode}
\cs_new_protected:Npn \@@_grab_l:w #1 \@@_run_code:
  { \@@_grab_l_aux:nN {#1} \cs_set_protected_nopar:Npn }
\cs_new_protected:Npn \@@_grab_l_long:w #1 \@@_run_code:
  { \@@_grab_l_aux:nN {#1} \cs_set_protected:Npn }
\cs_new_protected:Npn \@@_grab_l_aux:nN #1#2
  {
    \tl_set:Nn \l_@@_signature_tl {#1}
    \exp_after:wN #2 \l_@@_fn_tl ##1##
      { \@@_add_arg:n {##1} }
    \l_@@_fn_tl
  }
%    \end{macrocode}
% \end{macro}
% \end{macro}
% \end{macro}
%
% \begin{macro}{\@@_grab_u:w}
% \begin{macro}{\@@_grab_u_long:w}
% \begin{macro}{\@@_grab_u_aux:nnN}
% 抓取到一组记号的过程相当简单:定义抓取器,然后收集。
%    \begin{macrocode}
\cs_new_protected:Npn \@@_grab_u:w #1#2 \@@_run_code:
  { \@@_grab_u_aux:nnN {#1} {#2} \cs_set_protected_nopar:Npn }
\cs_new_protected:Npn \@@_grab_u_long:w #1#2 \@@_run_code:
  { \@@_grab_u_aux:nnN {#1} {#2} \cs_set_protected:Npn }
\cs_new_protected:Npn \@@_grab_u_aux:nnN #1#2#3
  {
    \tl_set:Nn \l_@@_signature_tl {#2}
    \exp_after:wN #3 \l_@@_fn_tl ##1 #1
      { \@@_add_arg:n {##1} }
    \l_@@_fn_tl
  }
%    \end{macrocode}
% \end{macro}
% \end{macro}
% \end{macro}
%
% \begin{macro}[EXP]{\@@_expandable_grab_u:w}
% 原来什么都不用做:接着是一个以函数命名的辅助功能,它完成了一切。
%    \begin{macrocode}
\cs_new_eq:NN \@@_expandable_grab_u:w \prg_do_nothing:
%    \end{macrocode}
% \end{macro}
%
% \subsection{访问参数规范}
%
% 有一段时间这是包含在内核中的,因此在本小节中我们必须始终使用\texttt{gset}(或类似的)。
%
% \begin{macro}{\@@_get_arg_spec_error:N, \@@_get_arg_spec_error:n}
% \begin{macro}{\@@_get_arg_spec_error_aux:n}
%   当尝试获取非\pkg{xparse}命令或环境的参数规范时提供信息性错误。
%    \begin{macrocode}
\cs_gset_protected:Npn \@@_get_arg_spec_error:N #1
  {
    \bool_set_false:N \l_@@_environment_bool
    \tl_set:Nn \l_@@_fn_tl {#1}
    \@@_get_arg_spec_error_aux:n { \cs_if_exist:NTF #1 }
  }
\cs_gset_protected:Npn \@@_get_arg_spec_error:n #1
  {
    \bool_set_true:N \l_@@_environment_bool
    \str_set:Ne \l_@@_environment_str {#1}
    \@@_get_arg_spec_error_aux:n
      { \cs_if_exist:cTF { \l_@@_environment_str } }
  }
\cs_gset_protected:Npn \@@_get_arg_spec_error_aux:n #1
  {
    #1
      {
        \msg_error:nne { cmd } { non-xparse }
          { \@@_environment_or_command: }
      }
      {
        \msg_error:nne { cmd } { unknown }
          { \@@_environment_or_command: }
      }
  }
%    \end{macrocode}
% \end{macro}
% \end{macro}
%
% \begin{macro}{\@@_get_arg_spec:NTF}
%   如果命令不是\pkg{xparse}命令,则报错。如果是,它的第二个\enquote{item}就是参数规范。
%    \begin{macrocode}
\cs_gset_protected:Npn \@@_get_arg_spec:NTF #1#2#3
  {
    \__kernel_cmd_if_xparse:NTF #1
      {
        \tl_set:Ne \ArgumentSpecification { \tl_item:Nn #1 { 2 } }
        #2
      }
      {#3}
  }
%    \end{macrocode}
% \end{macro}
%
% \begin{variable}{\ArgumentSpecification}
%    \begin{macrocode}
\tl_clear_new:N \ArgumentSpecification
%    \end{macrocode}
% \end{variable}
%
% \begin{macro}{\@@_get_arg_spec:N}
% \begin{macro}{\@@_get_arg_spec:n}
%   现在恢复参数规范就很简单了。
%    \begin{macrocode}
\cs_gset_protected:Npn \@@_get_arg_spec:N #1
  {
    \@@_get_arg_spec:NTF #1 { }
      { \@@_get_arg_spec_error:N #1 }
  }
\cs_gset_protected:Npn \@@_get_arg_spec:n #1
  {
    \exp_args:Nc \@@_get_arg_spec:NTF
      { environment~ \tl_to_str:n {#1} }
      { }
      { \@@_get_arg_spec_error:n {#1} }
  }
%    \end{macrocode}
% \end{macro}
% \end{macro}
%
% \begin{macro}{\@@_show_arg_spec:N}
% \begin{macro}{\@@_show_arg_spec:n}
%   显示参数规范简单来说就是找到它然后调用 \cs{tl_show:N} 函数。
%    \begin{macrocode}
\cs_gset_protected:Npn \@@_show_arg_spec:N #1
  {
    \@@_get_arg_spec:NTF #1
      { \tl_show:N \ArgumentSpecification }
      { \@@_get_arg_spec_error:N #1 }
  }
\cs_gset_protected:Npn \@@_show_arg_spec:n #1
  {
    \exp_args:Nc \@@_get_arg_spec:NTF
      { environment~ \tl_to_str:n {#1} }
      { \tl_show:N \ArgumentSpecification }
      { \@@_get_arg_spec_error:n {#1} }
  }
%    \end{macrocode}
% \end{macro}
% \end{macro}
%
% \begin{macro}{\GetDocumentCommandArgSpec}
% \begin{macro}{\GetDocumentEnvironmentArgSpec}
% \begin{macro}{\ShowDocumentCommandArgSpec}
% \begin{macro}{\ShowDocumentEnvironmentArgSpec}
%   简单映射,需要检查参数是否为单个控制序列或活动字符。
%    \begin{macrocode}
\cs_gset_protected:Npn \GetDocumentCommandArgSpec #1
  {
    \@@_check_definable:nNT {#1} \GetDocumentCommandArgSpec
      { \@@_get_arg_spec:N #1 }
  }
\cs_gset_eq:NN \GetDocumentEnvironmentArgSpec \@@_get_arg_spec:n
\cs_gset_protected:Npn \ShowDocumentCommandArgSpec #1
  {
    \@@_check_definable:nNT {#1} \ShowDocumentCommandArgSpec
      { \@@_show_arg_spec:N #1 }
  }
\cs_gset_eq:NN \ShowDocumentEnvironmentArgSpec \@@_show_arg_spec:n
%    \end{macrocode}
% \end{macro}
% \end{macro}
% \end{macro}
% \end{macro}
%
%    \begin{macrocode}
\msg_set:nnnn { cmd } { non-xparse }
  { \str_uppercase:n #1~not~defined~using~xparse. }
  {
    You~have~asked~for~the~argument~specification~for~the~#1,~
    but~this~was~not~defined~using~xparse.
  }
\msg_set:nnnn { cmd } { unknown }
  { Unknown~document~#1. }
  {
    You~have~asked~for~the~argument~specification~for~the~#1,~
    but~it~is~not~defined.
  }
%    \end{macrocode}
%
%    \begin{macrocode}
%</package>
%    \end{macrocode}
%
% \end{implementation}
%
% \PrintIndex
%
% \addtocontents{toc}{\protect\end{multicols}}
%